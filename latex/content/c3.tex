\chapter{Grundlagen}

\section{Einführung in cFront}

\section{CIA-Triade}

\section{Arten der Authentifizierung}
\begin{itemize}
    \item Die Authentifizierung dient häufig als erste Verteidigungslinie von Systemen. \cite{boonkrong2012security}
    \item Faktor Something you know. Diese Methode nutzt Informationen, welche nur dem Nutzer bekannt sind, um seine Identität zu bestätigen \cite{boonkrong2012security}.
    \item Faktor Something you have. Diese Methode nutzt physische Objekte, welche sich im Besitz des Nutzers befinden, um seine Identität zu bestätigen. Dazu gehören u.a. Smartcards und Hardware-Token \cite{boonkrong2012security}.
    \item Faktor Something you are. Diese Methode nutzt biometrische Daten des Nutzers, um seine Identität zu bestätigen. Dazu gehören u.a. Fingerabdrücke, Iris-Scans und Gesichtserkennung \cite{boonkrong2012security}.
    \item Ein Problem dieser Methode ist, dass sich menschliche Eigenschaften im Laufe der Zeit verändern können. Auch Verletzungen oder Krankheiten können die biometrischen Daten verändern \cite{boonkrong2012security}.
    \item Nicht standardmäßig, aber weiterer Faktor ist something you perform or produce. Diese Methode nutzt beispielsweise die Stimme oder die (digitale) Unterschrift des Nutzers, um seine Identität zu bestätigen \cite{boonkrong2012security}.
\end{itemize}


\section{Passwortbasierte Authentifizierung}

\begin{itemize}
    \item Die heutzutage am häufigsten genutzte Methode zur Authentifizierung ist die passwortbasierte Authentifizierung \cite{chanda2016password} \cite{boonkrong2012security} \cite{yildirim2019encouraging}.
    \item Die Sicherheit von Systemen basiert somit auf der Sicherheit der Passwörter \cite{boonkrong2012security}.
    \item Passwörter gelten als eins der größten Risiken für Systeme, da sie viele Angriffsvektoren bieten \cite{yildirim2019encouraging}.
    \item Obwohl es bereits alternative Ansätze gibt, werden Passwörter weiterhin genutzt. Das liegt an der Einfachheit und dem geringen Aufwand, welche die Nutzung von Passwörtern mit sich bringt \cite{yildirim2019encouraging}.
    \item Eine Vielzahl an großen Unternehmen wurden bereits Opfer von der Veröffentlichung von Passwörtern, obwohl ein hoher Aufwand betrieben wird diese zu schützen. Da sich die Enthüllung der Passwörter allerdings als Angriffsziel bei Angreifern etabliert hat, ist selbst ein hoher Aufwand nicht ausreichend \cite{boonkrong2012security}.
    \item Dabei handelt es sich am häufigsten um alphanumerische Passwörter, welche aus einer Kombination von Groß- und Kleinbuchstaben, Zahlen und Sonderzeichen bestehen \cite{chanda2016password}. 
    \item Passwörter können durch verschiedene Angriffe kompromittiert werden. Angreifer können Zugriff auf die Datenbank erhalten, in welcher die Passwörter gespeichert sind. Aber auch auf persönlicher Ebene können Passwörter erlangt werden. Aufgeschriebene Passwörter können in fremde Hände geraten. Auch Social Engineering kann genutzt werden, um Passörter mit Hilfe von Phishing oder Keyloggern zu erlangen. Häufig lassen sich Passwörter allerdings auch mit Hilfe von Brute-Force- oder Dictionary-Attacken kompromittieren \cite{chanda2016password}.
    \item Brute-Force-Attacken versuchen alle möglichen Kombinationen von Zeichen, welche ein Passwort enthalten kann, auszuprobieren. Je höher dabei die Anzahl an möglichen Kombinationen ist, desto aufwändiger wird es ein Passwort zu erraten.
    \item Je länger ein Passwort, desto schwieriger zu knacken. Länge auch wichtiger als Zeichenraum \cite{chanda2016password}.
    \item Hier auch kurz auf die Mathematik dahinter eingehen.
    \item Studien zeigen, dass Nutzer dazu neigen gleiche oder ähnliche Passwörter für verschiedene Zugänge zu nutzen \cite{chanda2016password} \cites{ives2004domino}.
    \item Verfügen Angreifer über ein Passwort eines Nutzers, können häufig auch andere Zugänge übernommen werden \cite{chanda2016password}.
    \item Obwohl die Angriffsvektoren und Schwachstellen von Passwörtern schon lange bekannt sind, bleiben diese unverändert bestehen. \cite{ives2004domino}.
\end{itemize}

\subsection{Speicherung}

\begin{itemize}
    \item Viele Angreifer versuchen Passwörter zu kompromittieren, indem sie Zugriff auf die Datenbank erhalten, in welcher die Passwörter gespeichert sind. Mit Hilfe von Passwörtern erhoffen sich die Angreifer Zugriff auf Systeme und Netzwerke \cite{boonkrong2012security}.
    \item Passwörter können auf verschiedene Arten gespeichert werden. Dadurch können verschiedene Angriffsvektoren entstehen \cite{chanda2016password}.
    \item Plaintext am schlechtesten. Werden die Passwörter in lesbarer Form gespeichert, können Angreifer alle Passwörter auslesen, sobald sie Zugriff auf die Datenbank haben. Dabei muss kein weiterer Aufwand betrieben werden \cite{chanda2016password}.
    \item Verschlüssellung besser, aber nicht optimal. Verschlüssellung ist zurückführbar. Gelangen Angreifer an den benötigten Schlüssel, können sie alle Passwörter entschlüsseln und auslesen \cite{chanda2016password}.
    \item AM besten Hashing mit Salt. Sobald ein Passwort gehasht wurde, kann es nicht mehr zurückgerechnet werden. Durch einen individuellen Salt kann ebenfalls verhindert werden, dass Angreifer die Passwörter mit Hilfe von Rainbow-Tables entschlüsseln können \cite{chanda2016password}.
    \item auch noch zwei salts möglich - einer public einer private. schützt vor offline angriffen \cite{chanda2016password}.
    \item Vielleicht hier noch ganz kurz auf Hash Funktionen eingehen?
\end{itemize}

\subsection{Faktor Mensch}

\begin{itemize}
    \item Die Sicherheit ist nicht nur von den technischen Aspekten abhängig \cite{ives2004domino}.
    \item Ein Großteil der Angriffsfläche von Passwörtern entsteht durch den Faktor Mensch \cite{yildirim2019encouraging}.
    \item Von Menschen erstellte Passwörter sind keine echten Zufallswerte. Das liegt insbesondere daran, dass Nutzer sich Passwörter merken können müssen. Daher beinhalten Passwörter häufig Informationen, welche einen Bezug zum Nutzer haben. Dazu gehören beispielsweise Namen, Geburtsdaten, Adressen oder andere persönliche Informationen. Auch Passwörter, welche einfache Tastaturmuster beinhalten sind sehr beliebt. Dazu zählen beispielsweise \glqq qwertz\grqq{} oder \glqq 123456\grqq{} \cite{chanda2016password} \cite{boonkrong2012security} \cite{yildirim2019encouraging}.
    \item Das Hauptproblem entsteht dabei durch die benötigte Einprägsamkeit der Passwörter \cite{yildirim2019encouraging}.
    \item Es ist sehr schwierig für Nutzer sich verschiedene komplexe Passwörter zu merken. Daher neigen Nutzer dazu, einfache Passwörter zu nutzen oder Passwörter für verschiedene Zugänge zu wiederholen \cite{chanda2016password}.
    \item Das ist der Hauptgrund dafür, dass Nutzer dazu neigen, einfache Passwörter zu nutzen oder Passwörter für verschiedene Zugänge zu wiederholen \cite{yildirim2019encouraging}.
    \item Diese Faktoren führen dazu dass die Anzahl an genutzten Passwörtern deutlich geringer ist als die gesamte Menge an möglichen Passwörtern \cite{boonkrong2012security}.
    \item Ebenfalls ist häufig die Motivation der Nutzer gering komplexe Passwörter zu erstellen. Dies liegt häufig daran, dass die Nutzer nicht die Gefahr erkennen und nicht überzeugt von Guidelines und Richtlinien zur Erstellung von Passwörtern sind \cite{yildirim2019encouraging}.
    \item Nutzer tendieren dazu bewusst schwache Passwörter zu erstellen, die den Anforderungen der Richtlinien entsprechen. Das führt zu einem kontraproduktiven Effekt, da die Sicherheit geringer wird \cite{yildirim2019encouraging}.
    \item Sehr komplexe Richtlinien führen demnach nicht zwangsmäßig zu einer höheren Sicherheit. Vielmehr kann das Gegenteil erreicht werden \cite{yildirim2019encouraging}.
    \item Aktive Internet-Nutzer verwalten durchschnittlich 15 Passwörter pro Tag \cite{ives2004domino}.
    \item Eine der größten Schwachstellen ist also die Wahl des Passwortes durch den Nutzer \cite{boonkrong2012security}.
    \item Ein Domino Effekt kann entstehen, wenn mit Hilfe eines Passwortes weitere Passwörter kompromittiert werden. So können mehrere Systeme indirekt davon betroffen sein, sobald ein Passwort kompromittiert wurde \cite{ives2004domino}.
    \item Das macht von Menschen erstellte Passwörter anfälliger für Angriffe, da diese einfacher zu erraten sind \cite{chanda2016password}.
    \item 
\end{itemize}

\section{Passwortlose Authentifizierung}

\subsection{Magic Link}

\subsection{One Time Password (OTP)}
\begin{itemize}
    \item Passwörter die sich mit jedem Login ändern. Dadurch wird das Risiko verringert, dass das Passwort erraten werden kann \cite{boonkrong2012security}.
\end{itemize}

\subsection{Biometrische Daten}

\subsection{Public Key Cryptography}

\section{Yubikey}

\begin{itemize}
    \item FIDO2 wird von der 
\end{itemize}

\section{Fido2}

\subsection{Webauthn}

\subsection{CTAP2}

\subsection{Sicherheit}