\chapter{Grundlagen}
Im Folgenden werden auf Basis der fachlichen Literaturrechere die benötigten Grundlagen dieser Arbeit beschrieben und dargestellt:

\section{Einführung in cFront}
\textit{cFront} ist eine Anwendung, welche von der Abteilung cGroup Solutions innerhalb der \ac{LSY} entwickelt wird. Es handelt sich bei der Anwendung um ein Tool, welches an Flughafen Terminals eingesetzt wird. 

Es agiert dabei als Interface für mehrere Anwendung und ermöglicht es dem Nutzer, diese Anwendungen zu starten. Welche Anwendungen sichtbar sind, können je nach Kunde individuell konfiguriert werden. 
\begin{figure}[h]
	\centering 
	\includegraphics[width=1\textwidth]{img/abbildungen/MicrosoftTeams-image (3).png}
	\captionsetup{format=hang}
	\caption{Umsetzungsmöglichkeit mit Keycloak}
\end{figure}

\section{CIA-Triade}

Die CIA-Triade gehört zu den wichtigsten Darstellungen von Sicherheitszielen innerhalb der Informationssicherheit. Sie beschreibt die drei Schutzziele \textit{Confidentiality} (Vertraulichkeit), \textit{Integrity} (Integrität) und \textit{Availability} (Verfügbarkeit). Im Folgenden werden diese kurz beschrieben:

\begin{figure}[h]
	\centering 
	\includegraphics[width=0.5
    \textwidth]{img/abbildungen/CIA-Triad.png}
	\captionsetup{format=hang}
	\caption{CIA-Triad} \label{CIA-Triad}
\end{figure}

\paragraph*{Confidentiality} 
Die Vertraulichkeit gehört zu den wichtigsten Schutzzielen in der Informationssicherheit \cite{samonas2014cia}. Das Wort \textit{Confedentiality} kommt vom lateinischen Wort \textit{confidere} und bedeutet so viel wie \textit{vertrauen} \cite{pons} \cite{samonas2014cia}. Das Schutzziel besagt, dass Informationen und Daten so geschützt sein müssen, dass diese nur von autorisierten Personen und für autorisierte Zwecke genutzt werden können \cite{samonas2014cia}. 
Dies beinhaltet beispielsweise Einschränkungen des Zugriffs auf Informationen und Daten, um die Privatsspähre und persönliches Eigentum zu schützen \cite{samonas2014cia}. Aber auch Verschlüsselungen, eine sichere Authentifizierung und Sicherheitsprotokolle können zur Gewährleistung der Vertraulichkeit beitragen \cite{agarwal2011security}.
Aufgrund der steigenden Wichtigkeit von wirtschaftlichen Aspekten hat die Vertraulichkeit im Vergleich zu früher an Bedeutung verloren \cite{samonas2014cia}. Häufig werden Schutzziele vernachlässigst, um im Gegenzug eine erhöhte Benutzerfreundlichkeit oder Wirtschaftlichkeit zu erreichen.

\paragraph*{Integrity}  
Das Wort Integrity leitet sich vom lateinischen Wort \textit{tangere} ab und bedeutet so viel wie \textit{berühren} \cite{pons}. Durch die Vorsilbe \textit{In-} soll eine Gegenteilige Bedeutung entstehen im Sinne von \textit{Unberührbarkeit} \cite{samonas2014cia}. 
Die Integrität soll somit garantieren, dass Daten nicht verändert werden können, ohne dass dies bemerkt wird \cite{agarwal2011security}. Schickt ein Sender $S$ beispielsweise eine Nachricht an einen Empfänger $E$, so soll die Nachricht identisch beim Empfänger $E$ ankommen, wie sie vom Sender $S$ gesendet wurde \cite{agarwal2011security}.
Umsetzungsmöglichkeiten die Integrität zu schützen beinhalten Maßnahmen wie beispielsweise das Verwenden einer Firewall, \ac{IDS} oder auch digitale Signaturen \cite{agarwal2011security}.
\paragraph*{Availability}
Das Wort Availability leitet sich vom lateinischen \textit{valere} ab und bedeutet so viel wie \textit{kräftig sein}. Die Verfügbarkeit bezieht sich also auf einen zeitnahen und und zuverlässigen Zugriff auf Informationen und Daten \cite{samonas2014cia}. Zuverlässig bedeutet dabei auch, dass ein Zugriff möglichst ohne Unterbrechungen und unabhängig vom Standort möglich ist \cite{agarwal2011security}.
Verfügbarkeit kann beispielsweise durch Netzwerksicherheit (z.B. Schutz vor \ac{DDoS}) oder Fehlertoleranz (z.B. durch Limitierung von Authentifizierungsversuchen) gewährleistet werden \cite{agarwal2011security}.

Ein wichtiger Aspekt der Schutzziele ist, dass diese nicht unabhängig voneinander betrachtet werden dürfen. Vielmehr handelt es sich, um ein Zusammenspiel der verschiedenen Schutzziele (siehe \ref*{CIA-Triad}). So kann eine Maßnahme beispielsweise mehrer Schutzziele schützen. Ebenfalls lassen sich weitere Schutzziele aus den drei bestehenden ableiten. Häufig werden erweiterte Schutzziele wie beispielsweise Authenticity (Authentizität) oder Non-repudiation (Nicht-Abstreitbarkeit) \cite{samonas2014cia} definiert. Diese können dabei zumeist von einem oder mehreren Schutzzielen der CIA-Triade abgeleitet werden. Die folgende Grafik zeigt beispielhaft einige erweiterte Schutzziele und deren Bezug zu der CIA-Triade:

\begin{figure}[h]
	\centering 
	\includegraphics[width=0.8\textwidth]{img/abbildungen/Schutzziele.png}
	\captionsetup{format=hang}
	\caption{Erweiterte Schutzziele} \label{Schutzziele}
\end{figure}


\section{Arten der Authentifizierung}

\begin{figure}[h]
	\centering 
	\includegraphics[width=0.4\textwidth]{img/abbildungen/factors.png}
	\captionsetup{format=hang}
	\caption{Faktoren der Authentifizierung}
\end{figure}

Die Authentifizerung dient häufig als erste Verteidigungslinie von Systemen \cite{boonkrong2012security}. Die Authentifizierung gilt als erweitertes Schutzziel und ist eine der wichtigsten Schutznmaßnahmen von Systemen. Sie übernimmt die Kontrolle über die Zugänge von Systemen und bestimmt wer oder was autorisiert ist diese zu nutzen. In der Fachliteratur wird häufig zwischen drei verschiedenen Arten der Authentifizierung unterschieden, welche umgangssprachlich auch als \textit{Faktoren} bekannt sind. Diese sollen im Folgenden beschrieben werden:

\paragraph*{Something you know:}
Die meistgenutzte Art der Authentifizierung basiert auf dem Wissen des Nutzers. Diese Methode nutzt Informationen - welche nur dem Nutzer bekannt sind - und bestätigt somit seine Identität \cite{boonkrong2012security}. Das bekannteste Verfahren ist dabei die Nutzung von Passwörtern, welche nur dem Nutzer bekannt sein sollten. Weitere Verfahren dieser Kategorie wären allerdings auch Sicherheitsfragen. Diese werden initial vom Nutzer beantwortet und im weiteren Verlauf zur Authentifizierung abgefragt.

\paragraph*{Something you have:}
Diese Art der Authentifizierung nutzt physische Objekte, um die Identität des Nutzers zu verifizieren. Es handelt sich um Objekte die sich lediglich im Besitz des Nutzers befinden \cite{boonkrong2012security}. Mögliche Beispiele für diese Methode sind Smartcards, welche an physische Zutrittskontrollen gehalten werden müssen oder Hardware Tokens, die für die Anmeldung an Systemen genutzt werden.

\paragraph*{Something you are:}
Diese Art der Authentifizierung basiert auf der Inhärenz. Das bedeutet, dass zur Verifizierung der Identität des Nutzers biometrische Merkmale verwendet werden \cite{boonkrong2012security}. Dazu gehören u.a. Fingerabdrücke, Geischtserkennung und Iris-Scans. Diese Methode hat sich besonders im Bereich der mobilen Systeme etabliert, so bietet Apple bei seinen Smartphones beispielsweise eine Authentifizierung per Fingerabdruck (\textit{Touch ID}) oder Gesichtserkennung (\textit{Face ID}) an CITE. Aber auch Microsoft bietet mittlerweile eine Authentifizierung mittels biometrischer Daten an (\textit{Windows Hello} und \textit{Hello for Business}) CITE.

Wie schon bei der CIA-Triade lassen sich auch hier weitere Arten ergänzen oder ableiten. Eine weitere Art ist beispielsweise \textbf{something you produce} \cite{boonkrong2012security}. Diese Art der Authentifizierung leitet sich teilweise von dem Faktor \textit{something you are} ab. Sie nutzt beispielsweise die Stimme des Nutzers oder seine (digitale) Unterschrift, um seine Identität zu verifizieren \cite{boonkrong2012security}. 

Die verschiedenen Arten der Authentifizierung spielen auch bei der Unterscheidung zwischen einer \ac{SFA} und einer  \ac{MFA} eine wichtige Rolle. Wird eine einzelner Faktor genutzt, so bezeichnet man dies als \ac{SFA}. Werden mehrere Faktoren genutzt handelt es sich um eine \ac{MFA}.

\section{Passwortbasierte Authentifizierung}\label{pw-auth}

    Die heutzutage am häufigsten genutzte Methode zur Authentifizierung ist die passwortbasierte Authentifizierung \cite{chanda2016password} \cite{boonkrong2012security} \cite{yildirim2019encouraging}. Diese basiert auf dem Faktor \textit{something you know}, also auf dem Wissen der Nutzer. Zumeist handelt es sich um alphanumerische Passwörter, welche aus einer Kombination von Groß- und Kleinbuchstaben, Zahlen und Sonderzeichen bestehen \cite{chanda2016password}. Die Sicherheit informationstechnischer Systeme ist somit abhängig von der Sicherheit der genutzten Passwörter \cite{boonkrong2012security}. Trotz ihrer weitreichenden Verbreitung gelten Passwörter als eine der größten Sicherheitsrisiken für Systeme, da sie vile Schwachstellen und Angriffsvektoren bieten \cite{yildirim2019encouraging} \cite{farke2020you}. Laut einer Studie von \textit{Verizon} basierten 2017 81\% der Hackerangriffe auf der Kompromittierung von Passörtern \cite{barbosa2021provable} \cite{verizon2017}. Eine weitere Studie zeigt auf, dass 2017 Phishing E-Mails die Angriffsmethode darstellte \cite{Symantec} \cite{barbosa2021provable}. Diese sind darauf ausgelegt an Passwörter von Nutzern zu gelangen. Eine Vielzahl von großen Unternehmen wurden bereits Opfer von der Veröffentlichung von Passwörtern, obwohl ein hoher Aufwand betrieben wird, um diese zu schützen \cite{boonkrong2012security}. Da sich die Enthüllung der Passwörter allerdings als Angriffsziel bei Angreifern etabliert hat, ist selbst ein hoher Aufwand nicht mehr immer ausreichend, um jene zu schützen \cite{boonkrong2012security}. Der entstehende Schaden ist immens, da es sich um einen hohen Geldwert, aber u.a. auch um einen Reputationsschaden handeln kann. Trotz der bekannten Schwachstellen und bereits entwickelten alternativen Ansätzen, bleibt das Passwort weiterhin genutzt \cite{ives2004domino}. Dies liegt insbesondere an der Einfachheit und dem geringen Aufwand, welche die Nutzung von Passwörtern mit sich bringt \cite{yildirim2019encouraging}.

    Passwörter können durch verschiedene Arten von Angriffen kompromittiert werden. So können Angreifer beispielsweise Zugriff auf die Datenbank erhalten, in welcher die Passwörter gespeicghert werden, aber auch auf persönlicher Ebene können Passörter erlangt werden. Dabei spielt das sog. Social Engineering eine große Rolle. Durch Shoulder Surfing können Angreifer versuchen Nutzern beim Passwort eintippen zuschauen. Mit Hilfe von Dumpster Diving können beispielsweise aufgeschriebene Passwörter erlangt werden. Zu den häufigsten Social Engineering Angriffen gehören allerdings die bereits beschriebenen Phishing Mails. Auf technischer Ebene ist ebenfalls ein Einsatz von Keyloggern möglich, welche alle Tastendrücke des Nutzers speichert. Ein häufig gewähltes und sehr effektives Mittel bei schlechten Passwörtern sind allerdings Brute-Force- und Dictionary-Angriffe. Diese kompromittieren Passwörter durch das stupide Ausprobieren aller möglichen Kombinationen oder die Nutzung von Tabellen, welche die meistgenutzten Passwörter beinhalten \cite{chanda2016password} \cite{morii2017research}.

    Um Passwörter resistenter gegen Brute-Force-Angriffe zu gestalten, kann eine Erweiterung des Zeichenraums oder der Passwort-Länge genutzt werden. So wird die mögliche Anzahl an Kombinationen des Passworts erhöht. Je mehr mögliche Kombinationen es gibt, desto schwieriger wird es Passwörter durch Erraten zu komprommittieren \cite{chanda2016password}. Wichtig ist hierbei, dass die Erweiterung der Passwortlänge deutlich effektiver ist als die Erweiterung des Zeichenraums. Betrachtet man die Anzahl aller Elemente des Zeichenraums $Z$ und die Passwortlänge $L$, so wird die Komplexität eines Passwortes durch $Z^L$ abgebildet. Während die Erweiterung der Passwortlänge ein exponentielles Wachstum aufweist, steigt bei einer Erweiterung des Zeichenraums die Steigung lediglich linear. Die Effektivität längerer Passwörter wird ebenfalls in \textbf{\ref{EntropyvsLength}} und \textbf{\ref{timetobreak}} dargestellt. \textbf{\ref{EntropyvsLength}} stellt die Entropie von Passwörtern in Abhängigkeit ihrer Länge dar. Dabei werden ebenfalls verschieden große Zeichenräume betrachtet. Es wird deutlich, dass selbst eine hohe Differenz des Zeichenraumes lediglich einen geringen Einfluss auf die Entropie hat. Unabhängig vom Zeichenraum aber dennoch eine hohe Entropie durch ein größere Länge möglich ist. \textbf{\ref{timetobreak}} stellt die benötigte Zeit zum Brechen von Passwörtern in Abhängigkeit zu ihrer Länge dar. Bei beiden Varianten handelt es sich um eine zu kurze Länge eines Passwortes, allerdings ist der signifikante Unterschied durch die Erweiterung der Passwortlänge um eins deutlich erkennbar.

    \begin{figure}[h]
        \centering 
        \includegraphics[width=0.7\textwidth]{img/abbildungen/entropy-length.png}
        \captionsetup{format=hang}
        \caption{Entropie in Abhängigkiet der Passwortlänge} \label{EntropyvsLength}
    \end{figure}
    
    \begin{figure}[H]
        \centering 
        \includegraphics[width=0.7\textwidth]{img/abbildungen/length-time.png}
        \captionsetup{format=hang}
        \caption{Zeit, um ein Passwort zu brechen in Abhängigkeit zu der Länge} \label{timetobreak}
    \end{figure}

    Zwei Angriffsvektoren sind dabei zumeist betroffen: die Speicherung und der Mensch. Im Folgenden soll präziser erläutert werden, was diese beiden Angriffsvektoren so verwundbar machen:

    \textbf{Speicherung:}

    Viele Angreifer versuchen Passwörter zu kompromittieren, indem sie Zugriff auf die Datenbank erhalten, in welcher die Passwörter gespeichert sind. Mit Hilfe der erlangten Passwörtern erhoffen sie sich zumeist einen erweiterbaren Zugriff auf Systeme oder nutzen die Passwörter, um ihre Opfer zu erpressen \cite{boonkrong2012security}. Der wichtigste Faktor für den Erfolg solcher Angriffe spielt die Art der Speicherung. Abhängig von der Art wie Passwörter gespeichert sind offenbaren sich auch verschiedene Schwachstellen \cite{chanda2016password}.
    Die schlechteste, aber dennoch immer noch genutzte Art Passwörter zu speichern ist die Speicherung von Passwörtern im Klartext. Die Passwörter werden also in lesbarer Form gespeichert. Haben Angreifer also Zugriff auf die Datenbank, so können sie alle gespeicherten Passwörter ohne weiteren Aufwand auslesen \cite{chanda2016password}.

    Eine bessere Variante - allerdings weitaus nicht optimale - ist die Verschlüssellung der gespeicherten Passwörter. Der größte Kritikpunkt an dieser Variante ist allerdings, dass Verschlüssellungen zurückführbar sind. Das bedeutet mit dem Besitz des benötigten Schlüssels, lassen sich alle gespeicherten Daten ebenfalls in Klartext umwandeln. Hierbei müssen Angreifer also einen weiteren Aufwand erbringen, um an den benötigten Schlüssel zu gelangen. Sind sie allerdings im Besitz dieses Schlüssels können sie ebenfalls alle gespeicherten Passwörter auslesen \cite*{chanda2016password}. 

    Um die Sepicherung weiter zu optimieren sollte somit keine Zurückführbarkeit bestehen. Dies kann mit Hilfe von Hashing umgesetzt werden. Sog. Hashfunktionen erhalten einen Eingabewert und bilden diesen auf (im Optimalfall) einen einzigen Ausgabewert ab. Dieser Ausgabewert ist nicht zurückführbar auf den Eingabewert. Komprommittieren Angreifer also die Datenbank, in welcher die Passwörter gespeichert sind, können diese die gespeicherten Werte nicht direkt weiterverwenden. Auch dieser Ansatz birgt allerdings Schwachstellen. So lassen sich beispielsweise sog. Rainbow-Tables nutzen, um Hash-Werte zurückzuführen. Dies wird ermöglicht indem häufig genutzte Passwörter gehashed werden und dann mit Hash-Werten innerhalb der Datenbank verglichen werden \cite{chanda2016password}. 

    Um auch diese Schwachstelle zu verhindern, wird ein sog. Salt benötigt. Dabei wird an jedes Passwort, bevor es gehashed wird, ein individueller randomisierter Wert gehangen. Somit wird verhindert, dass sich der gespeicherte Hashwert mit Hilfe von Rainbow-Tables vergleichen lässt. Auch eine Umsetzung mit zwei Salt-Werten ist möglich. Dabei ist ein Salt öffentlich und der andere privat. So kann ebenfalls ein Schutz gegen offline-Angriffe geboten werden \cite{chanda2016password}.

\textbf{Faktor Mensch:}

Neben den aufgezählten technischen Aspekten, stellt der Mensch selbst eine der größten Angriffsvektoren bezogen auf Passwörter dar \cite{ives2004domino} \cite{yildirim2019encouraging}. Eins der größten Problem stellt der Aspekt dar, dass von Menschen erstellte Passwörter keine echten Zufallswerte sind. Das liegt insbesondere daran, dass Nutzer sich ihre Passwörter merken müssen. Je komplexer ein Passwort gestaltet ist, desto schwieriger wird es für Nutzer sich dieses zu merken - insbesonere, wenn sie sich mehrere verschiedene Passwörter merken müssen. Daher beinhalten Passwörter häufig Informationen, welche einen Bezug zum Inhaber haben. Dazu gehören Beispielsweise Namen, Geburtsdaten, Adressen, oder andere persönliche Informationen. Auch Passwörter, welche einfache Muster beinhalten sind sehr beliebt. Dazu gehören beispielsweise \textit{qwertz}, welches die ersten Buchstaben auf der Tastatur darstellt und \textit{123456}. Solche Passwörter können sich Menschen besser einprägen, was notwendig ist, wenn Passwörter häufig genutzt werden müssen. Aus dem identischen Grund neigen Nutzer ebenfalls dazu ein Passwort für mehrere Systeme zu nutzen \cite{chanda2016password} \cite{boonkrong2012security} \cite{yildirim2019encouraging}. 

Die genannten Faktoren führen dazu, dass die Anzahl an genutzten Kombinationen für ein Passwort deutlich geringer ist als die gesamte Menge an möglichen Kombinationen \cite{boonkrong2012security}. Das macht von Menschen erstellte Passwörter deutlich anfälliger für Angriffe, da diese einfacher zu erraten sind \cite{chanda2016password}. Dies liegt häufig auch daran, dass die Motivation der Nutzer häufig gering ist, komplexe Passwörter zu erstellen, weil sie sich der Gefahr von schwachen Passwörtern nicht bewusst sind \cite{yildirim2019encouraging}. Kontraproduktiv wirken in diesem Zusammenhang auch Policies und Richtlinien zur Erstellung von Passwörtern \cite{yildirim2019encouraging}. Sind die Richtlinien zur Erstellung von Passwörtern zu komplex, tendieren Nutzer bewusst dazu Muster in das Passwort einzubauen, um sich dieses zu merken. Dies führt zu einem gegenteiligen Effekt, da die Sicherheit und die Komplexität der Passwörter dadurch sinkt. Die These, dass solche Richtlininien zwangsweise zu einer erhöhten Sicherheit beitragen ist somit ein Irrglaube \cite{yildirim2019encouraging} \cite{morii2017research}.

Ein weiteres Problem stellt die die bereits genannte mehrfache Nutzung eines Passwortes für verschiedene Systeme dar. Aktive Internet-Nutzer verwalten im Durchschnitt 15 Passwörter pro Tag \cite{ives2004domino}. Um sich also das Einprägen verschiedener Passwörter zu ersparen, wählen Nutzer tendenziell lieber ein Passwort. Das führt häufig zu einem Domino-Effekt im Falle einer Passwort-Kompromittierung. Gelangen Angreifer an ein einzelnes Passwort des Nutzers, ist es häufig möglich mit diesem auch Zugriff auf andere Systeme zu gelangen \cite{ives2004domino} \cite{morii2017research}. 


\section{Passwortlose Authentifizierung}

Unter dem Sammelbegriff der passwortlosen Authentifizierung werden verschiedene Verfahren zusammengefasst, welche die Nutzung von Passwörten ersetzen sollen. Im Gegensatz zur passwortbasierten Authentifizierung steht also nicht mehr der Faktor \textit{something you know} im Vordergrund, da das Wissen des Nutzers nicht mehr die Grundlage zur Verifizierung seiner Identität darstellen soll. Die \ac{FIDO} Allianz nutzt den Begriff passwortlose Authentifizierung beispielsweise, um eine eine \ac{SFA} oder \ac{MFA} mit der Hilfe eines Security Keys zu beschreiben \cite{farke2020you}. 

Passwortlose Verfahren werden dabei als sicherer im Vergleich zur passwortbasierten Alternative angesehen, da viele der in \textbf{\ref{pw-auth}} aufgeführten Angriffsvektoren passwortlosen Ansätze nicht existieren \cite{chowhan2019password} \cite{parmar2022comprehensive}. Zudem erhofft man sich eine zusätzliche erhoffte Benutzerfreundlichkeit durch passwortlose Verfahren - insbesondere, weil Nutzer sich keine Passwörter mehr merken müssen und so ein geringerer Aufwand besteht \cite{chowhan2019password}.

Passwortlose Verfahren haben sich jedoch noch nicht flächendeckend durchgesetzt und sind nicht annähernd so weit verbreitet wie die Nutzung von Passwörtern. Dies lässt sich auf mehrere Faktoren zurückführen. Häufig genannte Gründe innerhalb der Fachliteratur sind die Umgewöhnung der Nutzer an eine neuartige Authentifizierung, welches als Hürde zur Etablierung der passwortlosen Verfahren angesehen wird. Aber auch zusätzliche entstehende Kosten durch die Integration der neuen Verfahren können eine Verbereitung ausbremsen \cite{chowhan2019password}. Ein detailierter Einblick in die Vor- und Nachteile in Bezug auf der Benutzerfreundlichkeit wird in \textbf{\ref{Yubikey}} gegeben.

Es gibt dabei eine Vielzahl an Möglichkeiten eine passwortlose AUthentifizierung umzusetzen. Eine der am häufigsten genutzen Varianten ist die Nutzung von Security Keys in Kombination mit FIDO2. Diese liegen im Fokus dieser Arbeit und werden in \textbf{\ref{Yubikey}} und \textbf{\ref{fido2}} genauer beschrieben. Dennoch sollen auch mögliche Alternativen kurz vorgestellt werden:


\begin{figure}[h]
	\centering 
	\includegraphics[width=0.7\textwidth]{img/abbildungen/magic_link.png}
	\captionsetup{format=hang}
	\caption{Beispielhafte Umsetzung eines Magic Links} \label{magiclink}
\end{figure}

\textbf{Magic Link:}

Bei einem Magic Link handelt es sich um eine
Authentifizierungsmöglichkeit, bei welcher Nutzer
lediglich ihren Benutzernamen oder ihre E-Mail-Adresse
zur Anmeldung angeben müssen. Anschließend erhält der Nutzer eine E-Mail mit einem dazugehörigen Link, welcher genutzt wird, um seine Identität zu verifizieren \cite{chowhan2019password} \cite{parmar2022comprehensive}.
Dieser Link beinhaltet einen Authentication Code, welcher im Hintergrund abgeglichen und validiert wird. Ist die Validierung erfolgreich wird der Nutzer authentifiziert und angemeldet. Nach der Anmeldung verliert der AUthentication Code seine Gültigkeit und somit auch der Link selbst \cite{chowhan2019password}. Der Ablauf des Verfahrens wird ebenfalls vereinfacht in \textbf{\ref{magiclink}} dargestellt.
Die Sicherheit dieses Verfahrens basiert dabei auf der Annahme, dass der Mail-Server bzw. der Zugang zum Account des Nutzers ausreichend geschützt ist. Ist diese Annahme nicht gegeben können sich auch andere Personen mit dem Link des eigentlichen Nutzers authentifizieren ohne autorisiert zu sein \cite{chowhan2019password}. 

\paragraph*{Vorteile:} Ein Passwort bleibt zwar in den meisten Fällen für den Zugriff auf den E-Mail Zugang notwendig, würde aber zumindest die Anzahl an benötigten Passwörtern für Nutzer reduzieren. Zudem handelt es sich bei einem Magic Link um eine sehr benutzerfreundliche und einfach verständliche Art der Authentifizierung \cite{parmar2022comprehensive}. Auch die Implementierung und die Kosten zur Instandshaltung sind verhältnismäßig gering einzuorden \cite{parmar2022comprehensive}.

\paragraph*{Nachteile:} Insbesondere im Unternehmenskontext kann die Nutzung von Spam-Filtern die Benutzerfreundlichkeit von Magic Links stark beeinträchtigen. So können beispielsweise die zugehörigen Mails fälschlicherweise als Spam klassifiziert werden oder eine erhöhte Wartezeit auf die E-Mail entstehen \cite{parmar2022comprehensive}. Auch im Bezug auf das Thema Sicherheit sind einige Aspekte fragwürdig. So hängt die Sicherheit des Verfahrens von der Sicherheit des Mail-Servers ab. Ist dieser nicht ausreichend geschützt, können Angreifer Zugriff auf die Mails erhalten und sich ebenfalls mit dem Link authentifizieren \cite{chowhan2019password}. Dies kann geschehen, ohne dass der Nutzer dies überhaupt bemerkt \cite{chowhan2019password}.

\textbf{\ac{OTP}:}

Das Konzept hinter \ac{OTP}s ähnelt dem des Magic Links. Nutzer geben ihre E-Mail-Adresse oder ihre Handynummer an (diese können ebenfalls einem Benutzernamen zugewiesen sein) und erhalten eine E-Mail/SMS, welche ein \ac{OTP} beinhaltet \cite{chowhan2019password} \cite{parmar2022comprehensive}. 
Dieses wird vom System abgeglichen und validiert. Ist die Validierung erfolgreich wird der Nutzer authentifiziert und angemeldet. Nach der Anmeldung verliert das \ac{OTP} seine Gültigkeit \cite{chowhan2019password}.
Häufig werden \ac{OTP}s allerdings nicht für eine oben beschrieben \ac{SFA} genutzt, sondern dienen als zusätzlicher Faktor für eine \ac{MFA} \cite{chowhan2019password}. So können beispielsweise Authenticator Apps zur Bereitstellung von \ac{OTP}s genutzt werden, um die etabliertere passwortbasierte Authentifizierung sicherer zu gestalten.
Im Gegensatz zu statischen, von Anwendern gewählten
Passwörtern sind \ac{OTP}s dynamisch erzeugt und haben nur
eine geringe Lebensdauer. So wird eine höhere
Sicherheit gewährleistet, da \ac{OTP}s nur schwierig durch
stupides Erraten oder Brute Force Attacken erbeutet
werden können \cite{chowhan2019password}.
Für die Umsetzung von OTPs gibt es mehrere Möglichkeiten. Zwei häufig verwendete Optionen sind \ac{HOTP} und \ac{TOTP} \cite{chowhan2019password}.
\ac{HOTP}s basieren auf der technischen Spezifikation RFC 4226. Sie werden mit Hilfe von \ac{HMAC} und unabhängig von der Zeit generiert. Neue \ac{HOTP}s können Event-basiert von dem Nutzer angefordert werden \cite{chowhan2019password}.
\ac{TOTP}s basieren auf der technischen Spezifikation RFC 6238 und werden in Abähngigkeit zu der Zeit erstellt. Sie ändern sich nach einem vordefinierten Zeitintervall und sind somit sehr kurzlebig \cite{chowhan2019password}.

\paragraph*{Vorteile:} Die Nutzung von \ac{OTP}s ist sehr effektiv für eine \ac{MFA}, da die Sicherheit im Vergleich zu einer passwortbasierten \ac{SFA} signifikant erhöht werden kann \cite{chowhan2019password} \cite{parmar2022comprehensive}. Zudem handelt es sich um eine sehr benutzerfreundliche und einfachz anwendbare Methode, welche sich bereits für die Nutzung von \ac{MFA} weitreichend etabliert hat \cite{parmar2022comprehensive}. Dies liegt auch an der Vielzahl an Umsetzungsmöglichkeiten von \ac{OTP}s, da diese beispielsweise via E-Mail, SMS, Authenticator App oder auch Security Key an den Nutzer übermittelt werden können \cite{chowhan2019password} \cite{parmar2022comprehensive}. 

\paragraph*{Nachteile:} Da sich diese Arbeit auf die Nutzung einer passwortlosen Authentifizierung als \ac{SFA} fokussiert, wird es hier als Nachteil eingeordnet, dass sich die Nutzung von \ac{OTP}s hauptsächlich für die Umsetzung einer \ac{MFA} anbietet. Eine Nutzung von \ac{OTP}s als \ac{SFA} wird häufig nicht unterstützt. Zudem ist je nach Implementierung wie auch bei einem Magic Link eine Abhängigkeit auf einen anderen Dienst gegeben, welche die Sicherheit des Verfahrens beeinträchtigen können.

\begin{itemize}
    
    \item Biometrische Daten:
    Eine häufig genutzte Methode zur Authentifizierung auf mobilen Endgeräten ist die Nutzung von biometrischen Daten \cite{parmar2022comprehensive}. 
    Hierbei werden einzigartige biometrische Merkmale des Nutzers genutzt um seine Identität zu verifizieren. Dazu gehören beispielsweise Fingerabdrücke oder eine Gesichtserkennung \cite{parmar2022comprehensive}. Diese Variante kann auch im Unternehmenskontext unter anderem mit Windows Hello for Business genutzt werden.
    Vorteile:
    Viele mobile Endgeräte arbeiten bereits mit biometrischen Daten \cite{parmar2022comprehensive}.
    Keine große Umgewöhnung für Nutzer, da diese oftmals bereits mit biometrischen Daten arbeiten \cite{parmar2022comprehensive}.
    nahezu einzigartig und somit deutlich schwieriger anzugreifen als passwörter \cite{parmar2022comprehensive}.
    Bereits für Unternehmenskontext verfügbar, beispielsweise mit Windows Hello for Business.
    Nachteile:
    Äußere Bedingungen können die Erkennung von biometrischen Daten beeinträchtigen. beispielsweise schlechtes licht bei Gesichtserkennung oder staubige Umgebungen bei Fingerabdruckscanner \cite{parmar2022comprehensive}.
    Biometrische Daten können sich im Laufe der Zeit verändern. Auch Verletzungen oder Krankheiten können die biometrischen Daten verändern \cite{boonkrong2012security}.
    Im Unternehmenskontext häufig verschiedene Hersteller und Geräte, welche nicht alle biometrischen Daten unterstützen oder nicht untereinander kompatibel sind \cite{parmar2022comprehensive}.

    \item FIDO2:
    WIRD IN KAPITEL GENAUWER BESCHRVIEBEN


\end{itemize}


\section{YubiKey} \label{Yubikey}

\begin{figure}[h]
	\centering 
	\includegraphics[width=0.3\textwidth]{img/abbildungen/yubikey.jpeg}
	\captionsetup{format=hang}
	\caption{Umsetzungsmöglichkeit mit Keycloak}
\end{figure}

 \begin{itemize}
    \item Ein Security Key ist eine Hardware, welche es ermöglicht einen Nutzer zu authentifizieren, indem dieser mit dem Security Key interagiert (beispielsweise durch einen Knopfdruck) \cite{reynolds2018tale}.
    \item Häufig werden Security Keys so designed, dass sie per USB an einen Computer angeschlossen werden können \cite{reynolds2018tale}.
    \item Die YubiKeys 5 ermöglichen drei Arten der Authentifizierung:
    1. \ac{SFA} Ersetzt Passwörter durch ein passwortloses \textit{tap-n-go} Verfahren.
    2. \ac{2FA} Sichert ein Passwort zusätzlich mit einem \textit{tap-n-go} Faktor ab. Der Security ist somit der zweite Faktor (\textit{something you have}).
    3. \ac{MFA} Verbindet die passwortlose \textit{tap-n-go} Authentifizierung mit einer PIN. (EIG AUCH SFA ODER NICHT)
 \end{itemize}

\subsection{Usability}

\begin{itemize}
    \item Vorteile:
    \item Ergebnisse zeigen, dass Nutzer grundsätzlich bereit sind, Passwörter durch passwortlose Verfahren zu ersetzen \cite{lyastani2020fido2}.
    \item Passwortlose Verfahren mit Yubikey wurden mehr akzeptiert als tradionelle passwortbasierte Verfahren \cite{lyastani2020fido2}.
    \item Implizite Garantie, dass sich lediglich Nutzer authentifizieren können, welche auch im Besitz des Authentifizierungsgerätes sind. \cite{lyastani2020fido2}.
    \item Durch die Nutzung von FIDO2 kann die Usability verbessert werden, da Nutzer sich keine Passwörter mehr merken müssen. Häufig wird das Verwalten der immer höher werdenden Anzahl an Passwörtern als Problem angesehen \cite{lyastani2020fido2} \cite{farke2020you}.
    \item Es wird ein deutlich geringerer kognitiver Aufwand benötigt, da Nutzer keine neuen Passwörter mehr erstellen und merken müssen \cite{lyastani2020fido2}.
    \item Zum aktuellen Zeitpunkt wird FIDO2 bereits von einer Vielzahl an Browsern unterstützt. Zusätzlich bieten immer mehr Online-Dienste die Möglichkeit an sich mit Hilfe von FIDO2 zu authentifizieren \cite{lyastani2020fido2} \cite{farke2020you}.
    \item Es handelt sich um offene und standardisierte Protokolle \cite{farke2020you}.
    \item Nachteile:
    \item Im Falle einer \ac{SFA} wird der Verlust des Authentifizierungsgerätes als größtes Problem angesehen. Bei Verlust hat auch der Nutzer keinen Zugriff mehr und aktuell gibt es noch keine sichere und effiziente Möglichkeiten, um den Zugriff wiederherzustellen (vor allem ohne Pause) \cite{lyastani2020fido2}.
    \item Da es sich um zusätzliche Hardware handelt kann diese ebenfalls kaputt gehen \cite{farke2020you}.
    \item Im Unternehmenskontext, kann die Verwaltung und Verteilung der Authentifizierungsgeräte zu einem Problem werden \cite{farke2020you}. 
    \item Da es sich um Hardware handelt, können Zugänge nicht an vertraute Personen weitergegeben werden, da der Zugang nur mit dem Authentifizierungsgerät möglich ist \cite{lyastani2020fido2}.
    \item Ohne das Authentifizierungsgerät sind keine spontanen Logins möglich \cite{lyastani2020fido2}.
    \item Es wird ein physischer Aufwand benötigt, da das Authentifizierungsgerät mitgeführt werden muss \cite{lyastani2020fido2}.
    \item Bereits das aus der Tasche holen des Authentifizierungsgerätes ist für manche Nutzer bereits eine Hürde \cite{farke2020you}.
    \item Authentifizierungsgeräte sind häufig mit Kosten verbunden, welche vom Nutzer getragen werden müssen \cite{lyastani2020fido2}.
    \item Nutzer haben Probleme ein neues Verfahren für die Authentifizierung zu nutzen, da sie sich an das alte Verfahren gewöhnt haben. Das führt dazu, dass Nutzer das neue Verfahren als kompliziert und ungewohnt empfinden. Sie verfügen häufig nicht über das nötige Wissen, um die Funktion und Sicherheit des Verafahrens zu verstehen \cite{lyastani2020fido2}.
    \item Selbst Nutzern, welchen das Konzept der passwortlosen Authentifizierung gefällt, nutzen häufig weiterhin Passwörter \cite{farke2020you}.
    \item Nutzer wollen keine Angewohnheiten verändern, wenn die nicht dazu gezwungen sind \cite{farke2020you}.
    \item Nutzer verwenden lieber Passwörter, da sie das Konzept und die Technologie besser verstehen \cite{lyastani2020fido2}.
    \item Nicht zwangsweise schneller als die Nutzung von Passwortmanagern \cite{farke2020you}.
    \item Allgemein fällt das Feeback von Nutzern weniger positiv aus, wenn diese vorher bereits Passwortmanager genutzt haben \cite{farke2020you}.
    \item Fazit:
    \item Insgesamt lassen sich noch nicht alle Szenarien mit FIDO2 abdecken. Es gibt noch spezielle Fälle, in welchen die Nutzung von Passwörtern weiterhin notwendig ist \cite{lyastani2020fido2}.
    \item 
\end{itemize}

\section{Fido2} \label{fido2}

\begin{figure}[h]
	\centering 
	\includegraphics[width=1\textwidth]{img/abbildungen/fido2-simple.png}
	\captionsetup{format=hang}
	\caption{Umsetzungsmöglichkeit mit Keycloak}
\end{figure}

\begin{figure}[h]
	\centering 
	\includegraphics[width=1\textwidth]{img/abbildungen/Fido2.png}
	\captionsetup{format=hang}
	\caption{Umsetzungsmöglichkeit mit Keycloak}
\end{figure}

\begin{figure}[h]
	\centering 
	\includegraphics[width=1\textwidth]{img/abbildungen/fido2_usability.png}
	\captionsetup{format=hang}
	\caption{Umsetzungsmöglichkeit mit Keycloak}
\end{figure}

\begin{itemize}
    \item FIDO2 wird von der \ac{FIDO} und dem \ac{W3C} entwickelt und bereitgestellt \cite{lyastani2020fido2} \cite{farke2020you}.
    \item Die \ac{FIDO} Allianz ist eine Organisation mit weltweit über 250 Mitgliedern. Darunter befinden sich Unternehmen wie Google, Microsoft, Apple Amazon, Facebook, Visa und viele mehr \cite{lyastani2020fido2} \cite{farke2020you}.
    \item Ziel ist es Nutzer zu authentifizieren, ohne, dass diese eine Passwort nutzen müssen \cite{morii2017research} \cite{barbosa2021provable}.
    \item Basiert auf der Nutzung eines internen oder externen Authentifizierungsgerätes \cite{morii2017research} \cite{barbosa2021provable}.
    \item Dabei können Authentifizierungsgeräte, ebenfalls mit einer PIN oder einem biometrischen Merkmal, geschützt werden \cite{farke2020you}.
    \item Hierbei ist ein PIN allerdings nicht gleichzusetzen mit einem Passwort. Der PIN wird lediglich für das Authentifizierungsgerät genutzt und wird auch nur auf diesem gespeichert \cite{farke2020you} \cite{barbosa2021provable}.
    \item Es handelt sich dabei also auch nicht um eine \ac{MFA}, sondern, um einen einzelnen Faktor, welcher lediglich den Zugriff das Gerät selbst authentifiziert \cite{barbosa2021provable}.
    \item FIDO2 unterstützt sowohl \ac{MFA} als auch \ac{SFA} \cite{lyastani2020fido2} \cite{farke2020you}.
    \item Viele Alternativen zur passwortbasierten AUthentifizierung existieren bereits. Diese werden allerdings nur in einem sehr geringen Ausmaß genutzt \cite{farke2020you}.
    \item Stellt Zugangsdaten bereit, welche nicht gephisht oder von Datenlecks betroffen sein können \cite{lyastani2020fido2}.
    \item Das liegt daran, dass keine geteilten Geheimnisse zwischen Nutzer und Dienst existieren, welche auf einem Server gespeichert werden \cite{morii2017research}.
    \item Wird von fast allen Browsern standardmäßig unterstützt \cite{lyastani2020fido2}.
    \item Viele verfügbare Authentifizierungsgeräte. Z.B. Security Keys oder auch Smartphones. Beispielsweise Apples Touch ID oder Face ID \cite{lyastani2020fido2}.
    \item Besteht aus zwei Komponenten: CTAP2 für die Kommunikation zwischen Client und Authentifizierungsgerät und WebAuthn für die Kommunikation zwischen Client und Server \cite{farke2020you}.
    \item Dabei wird WebAuthn von der \ac{W3C} spezifiziert und CTAP2 von der \ac{FIDO} Allianz \cite{farke2020you}.
\end{itemize}

\subsection{Webauthn}

\begin{itemize}
    \item WebAuthn ist ein Standard, welcher von dem \ac{W3C} entwickelt wird. Das Protokoll erlaubt es Webanwendungen Nutzer zu authentifizieren. Dies kann dabei auch über \ac{CTAP2} erfolgen \cite{lyastani2020fido2}. ?
    \item Wurde 2019 ein offizieller Webstandard \cite{farke2020you}.
    \item Spezifiziert eine standardiserte, vom Browser unabhängige JavaScript API zur Authentifizierung von Nutzern für Webanwendungen. So können Webanwendungen eine Authentifizierung integrieren, welche resistent gegenüber Phishing, Datenlecks und Passwortdiebstahl ist. Anstelle von geteilten Geheimnissen nutzt WebAuthn public-key Kryptographie, um einzigartige Zugangsdaten für jede Webanwendung zu erstellen, welche nur auf dem Gerät des Nutzers gespeichert werden \cite{farke2020you}.
    \item Passwortloses Challenge-Response-Verfahren zwischen Client und Server \cite{barbosa2021provable}.
    \item WebAuthn unterstützt zwei Opertationen: Registrierung und Anmeldung \cite{barbosa2021provable}.
    \item In der Registrierungsphase sendet der Server dem Authentifizierungsgerät über den CLient eine zufällige Challenge. In dieser Phase signiert das Authentifizierungsgerät mit Hilfe seines privaten Schlüssels die CHallenge und sendet zusätzlich  öffentliche Anmeldedaten für zukünftige Anmeldungen an den Server. Meldet sich ein bereits registrierter Nutzer an, wird die Challenge des Servers erneut von dem Authentifizierungsgerät signiert zurück an den Server gesendet. Der Server kann die Signatur mit Hilfe des öffentlichen Schlüssels verifizieren und den Nutzer authentifizieren \cite{barbosa2021provable}.
    \item Registrierungsphase:
    Der Server \textit{S} sendet eine challenge message $m_{rch}$ über den Client \textit{C} an den Security Key. Diese Challenge beinhaltet eine randomisierte Nonce, Parameter (beispielsweise, ob eine Nutzerverifizierung notwendig ist) und optional einen wert \textit{tb}, welcher den zugrunde liegenden Kanal eindeutig identifiziert (typischerweise eine \ac{TLS} Verbindung). Der Client \textit{C} erhält die challenge message $m_{rch}$ und wandelt diese in eine command message $m_{rcom}$ und eine client mesasage $m_{rcl}$ um. die command message $m_{rcom}$ wird an den Security Key $T$ übermittelt. Der Security Key $T$ erzeugt öffentlich-privates Schlüsselpaar, welches an den Server $S$ gebunden ist und diesem ermöglicht eine Verifizierung, während der folgenden AUthentifizierungsphase durchzuführen. Zudem gibt der Security Key $T$ eine response message $m_{rrsp}$ aus. Der Client übergibt diese und die client message $m_{rcl}$ an den Server $S$. Die response message $m_{rrsp}$ beinhaltet einen \textit{attestation type}, welcher es dem Server $S$ ermöglicht eine Verifizierung während der Registrierungsphase durchzuführen und beinhaltet den öffentlichen Schlüssel. WebAuthN 2 unterstützt fünf \textit{attestation types}. Häufig werden die types \textit{None} und \textit{Basic} verwendet. Die restlichen types sind \textit{Self}, \textit{AttCA} und \textit{AnonCA}. \cite{bindel2022fido2}
    \item Authentifizierungsphase:
    Der Client empfängt die challenge message $m_{ach}$ von Server $S$ und wandelt diese in eine command message $m_{acom}$ und eine client message $m_{acl}$ um. Die command message $m_{acom}$ wird an den Security Key $T$ übermittelt. Der Security Key $T$ erzeugt eine response message $m_{arsp}$, welche mit dem privaten Schlüssel signiert wird und sendet diese an den Server $S$ (über den Client $C$). Der Server $S$ akzeptiert die response message $m_{arsp}$ und die client message $m_{acl}$ nur, wenn sie sich mit dem dazugehörigen öffentlichen Schlüssel verifizieren lassen. \cite{bindel2022fido2}
    \item Die Sicherheit von WebAuthn basiert auf dem Beweis, dass RSASSA PKCS1-v1\_5 und RSASSA-PSS als \ac{EUF-CMA} gelten und der Annahme, dass SHA-256 kollisionsresistent ist \cite{barbosa2021provable}.
\end{itemize}

\subsection{CTAP2}
\begin{itemize}
    \item 2018 wurde \ac{CTAP2} als internationaler Standard der \ac{ITU-T} anerkannt \cite{barbosa2021provable}.
    \item \ac{CTAP2} ist ein Protokoll auf der Anwendungsebene, welches für die Kommunikation zwischen eines WebAuthn Clients und eines konformen Authentifizierungsgerätes genutzt wird. Das Authentifizierungsgerät kann dabei ein externes Gerät sein wie beispielsweise ein Security Key, welches über USB, Bluetooth oder NFC eine Verbindung mit dem Client aufbaut. Aber auch ein internes Gerät wie beispielsweise ein Fingerabdruckscanner oder ein Trusted Platform Module können als Authentifizierungsgerät genutzt werden \cite{lyastani2020fido2}.
    \item \ac{CTAP2} spezifiziert, die Kommunikation zwischen einem Authentifizierungsgerät und einem Client. Der Client ist dabei üblicherweise ein Webbrowser. Das Ziel ist es zu garantieren, dass der CLient das AUthentifizierungsgerät nur nutzen darf, wenn der Nutzer dies erlaubt. Dafür muss der Nutzer beispielsweise einen Knopf am Authentifizierungsgerät drücken und/oder sich mit Hilfe eines PINs oder eines biometrischen Merkmals beim AUthentifizierungsgerät authentifizieren \cite{barbosa2021provable}.
    \item Das Ziel ist es somit einen Client an das AUthentifizierungsgerät zu binden. Ist ein CLient nicht an das authentifizerungsgerät gebunden, kann dieser sich nicht authentifizieren \cite{barbosa2021provable}.
    \item besteht aus mehreren Phasen. 
    1. In der Setup Phase initialisiert ein Client $C'$ einen PIN, welcher vom User übergeben wird an den Security Key $T$.
    2. In der Binding Phase tauschen ein Client $C$ (nicht zwangsweise $C'$) und der Security Key $T$ einen gemeinsamen Verbindungsstatus aus, wenn der Client $C$ in der Lage ist, Informationen über die auf dem Security Key $T$ gespeicherte PIN zu liefern. So soll eine einzigartige Verbindung zwischen dem Client $C$ und dem Security Key $T$ hergestellt werden. Schlägt der Client $C$ drei mal in Folge fehl die PIN zu liefern, wird der Security Key $T$ neu gestartet und der Verbindungsstatus wird zurückgesetzt. Schlägt der Client $C$ insgesamt acht mal fehl, wird der Security Key $T$ gesperrt.
    3. Ist diese Phase erfolgreich, autorisiert der Client $C$ jeden Befehl, indem eer einen Tag $t$ ausgibt, welcher mit der command message an den Security Key $T$ übermittelt wird. Der Security Key $T$ fährt lediglich fort, wenn eine \textit{positive decision} $d$ des Users vorliegt (beispielsweise einem Knopfdruck) und validiert darauf hin die command message und den Tag $t$. \cite{bindel2022fido2}
    \item \ac{CTAP2} nutzt unauthentifizierten Diffie-Hellman Schlüsselaustausch \cite{barbosa2021provable}.
    \item Dieser kann von \ac{MITM} Angriffen betroffen sein \cite{barbosa2021provable}.
    \item In der Binding Phase sendet das Authentifizierungsgerät dem CLient ein pinToken, welcher beim hochfahren des Authentifizierungsgerätes generiert wird. Dieser pinToken wird lokal auf dem AUthentifizierungsgerät gespeichert und wird von dem verbundenen Client in der Access Channel Phase genutzt, um die nachfolgenden Nachrichten des Clients zu autorisieren \cite{barbosa2021provable}.
    \item Jedem AUthentifizierungsgerät wird ein pinToken pro hochfahren zugeordnet. Das bedeutet mehrere CLients erezugen mehrere Access Channels mit dem selbem Authentifizierungsgerät und dem selben pinToken \cite{barbosa2021provable}.
    \item Dadurch wird die Sicherheit von \ac{CTAP2} limitiert ??
\end{itemize}

\subsubsection{CTAP2.1}
\begin{itemize}
    \item Gilt in Verbindung mit WebAuthn 2 als \ac{PQ} bereit, da ein Operationsmodus ermöglicht wird, der nur kryptographische Primitive, digitale Signaturen und \ac{KEM} verwendet \cite{bindel2022fido2}.
    \item Im Gegensatz zu \ac{CTAP2} basiert CTAP2.1 nicht auf unauthentifizierten Diffie-Hellman Schlüsselaustausch, sondern auf einem sogeannten \ac{puvProtocol}, wodurch die \ac{PQ}-Sicherheit ermöglicht wird \cite{bindel2022fido2}.
    \item In \ac{CTAP2} wird der Verbindungszustand als \textit{pinToken} definiert, welcher aus mehreren 128 Bit-Blöcken besteht und keine maximale Begrenzung der Länge beseitzt. In CTAP2.1 wird der Verbindungszustand als \textit{pinUvAuthToken} definiert welcher eine feste Länge von 128 oder 256 Bit besitzt \cite{bindel2022fido2}.
    \item Der pinToken von \ac{CTAP2} wird bis zum nächsten Neustart wiederverwendet. Der pinUvAuthToken von CTAP2.1 wird nach jeder erfolgreichen Authentifizierung neu generiert. Das führt dazu, dass CTAP2.1 eine \ac{SUF}-t' Sicherheit aufweist und \ac{CTAP2} lediglich eine \ac{UF}-t' Sicherheit \cite{bindel2022fido2}.
    \item \ac{CTAP2} erlaubt es Security Keys und Clients nur den pinUvAuthToken zu teilen, wenn der Nutzer den korrekten PIN eingegeben hat. CTAP2.1 ermöglicht zusätzlich, dass der Nutzer sich mit Hilfe eines biometrischen Merkmals authentifiziert \cite{bindel2022fido2}.
\end{itemize}

\subsection{Sicherheit}

\begin{itemize}
    \item FIDO2 ist eine Erweiterung des FIDO U2F Protokolls und bietet die selbe Sicherheit wie public key Kryptographie \cite{lyastani2020fido2}.
    \item Es handelt sich um geprüfte assymetrische Kryptographie \cite{farke2020you}.
    \item Es handelt sich dabei um ein Challenge-Response-Verfahren mittels Hardware basierten Authentifizierungsgeräten. Dies bietet einige Vorteile gegenüber passwortbasierten Verfahren. Es gibt keine geteilten Geheimnisse zwischen Usern und Diensten, welche durch Phishing oder Datenlecks kompromittiert werden können. Dabei ist das selbe Authentifizierungsgerät für mehrere Dienste nutzbar, ohne, dass sich dabei eine Verknüpfung zurückführen lässt \cite{lyastani2020fido2} \cite{farke2020you}.
    \item lediglich die Session kann kompromittiert werden \cite{morii2017research}.
    \item Authentifizierungsgeräte lassen sich mit zusätzlichen PINs oder biometrischen Merkmalen absichern, um sich ebenfalls vor Diebstahl schützen \cite{barbosa2021provable}.
    \item Unauthentifizierter Diffie-Hellman Schlüsselaustausch könnte durch ein \ac{PAKE} Verfahren ersetzt werden \cite{barbosa2021provable}.
    \item Das Paper gibt an, dass dieses sicherer und effizienter sein soll \cite{barbosa2021provable}.
    \item In folgendem Szenario: 1. Der Nutzer besitzt einen Security Key, welcher mit einem drückbaren Knopf oder ähnlichen ausgestattet ist. 2. Der Security Key ist mit einem geheimen PIN geschützt. 3. Der Nutzer autorisiert vertrauten Clients auf den Security Key zuzugreifen. 4. Der Nutzer verbindet seinen Security Key mit mehreren Clients und nutzt diese um sich bei mehreren Webdiensten zu registrieren/anzumelden.
    Dann ist versichert, dass:
    1. Die Authentifizierung von dem Security Key durchgeführt wurde, welcher die genutzen Zugangsdaten bei dem Webdienst registriert hat.
    2. ein autorisierter Befehl auf den Security Key zugegriffen hat.
    3. und dieser autorisierte Befehl von einem autorisierten Client beauftragt wurde (sollte der Nutzer den korrekten PIN eingegeben haben).
    Dies setzt voraus, dass:
    1. Der Security Key nicht gestohlen wurde.
    2. Der PIN des Security Keys nicht kompromittiert wurde.
    3. Der autorisierte Client nicht kompromittiert wurde (korrekte Ausführung von \ac{CTAP2} und CLient ist nicht von böswilliger Software betroffen).
    \cite{barbosa2021provable}.
    \item Wird ein Security Key gestohlen, kann dieser nur genutzt werden, wenn ebenfalls der PIN bekannt ist \cite{barbosa2021provable}.
\end{itemize}
