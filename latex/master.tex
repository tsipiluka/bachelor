%
%   Prof. Dr. Julian Reichwald
%   auf Basis einer Vorlage von Prof. Dr. Jörg Baumgart
%   DHBW Mannheim
%
%
%	ACHTUNG: Für das Erstellen des Literaturverzeichnisses wird das modernere Paket biblatex
%			 in Kombination mit biber verwendet -- nicht mehr das ältere BibTex!
% 			 Bitte stellen Sie ggf. Ihre TeX-Umgebung
% 			 entsprechend ein (z.B. TeXStudio: Einstellungen --> Erzeugen --> Standard Bibliographieprogramm: biber)
%

\documentclass[
	12pt,
	BCOR=5mm,
	DIV=10,
	headinclude=on,	
	footinclude=off,
	parskip=half,
	bibliography=totoc,
	listof=entryprefix,
	toc=listof,
	pointlessnumbers
	]{scrreprt}

%	Konfigurationsdatei einziehen
\input{config}

\begin{document}



\TitelDerArbeit{Analyse und Integration einer passwortlosen Authentifizierung im Unternehmenskontext}
\AutorDerArbeit{Luka Tsipitsoudis}
\Firma{Lufthansa Systems GmbH \& Co. KG}
\Kurs{TINF20CS1}

\definecolor{tabdavid}{RGB}{213,229,255}

\pagenumbering{Alph}
\setcounter{page}{1} 

\begin{titlepage}
    \begin{minipage}{\textwidth}
            \vspace{-2cm}
            \noindent
            \includegraphics[scale=0.035]{img/Uzletiforditas.hu-Lufthansa.png} 
             \hfill   
             \includegraphics{img/logo.jpg}
    \end{minipage}
    
    \vspace{2em}
    \sffamily
    \begin{center}
        \textsf{\large{}Duale Hochschule Baden-W\"urttemberg\\[1.5mm] Mannheim}\\[2em]
        \textsf{\textbf{\Large{}Bachelorarbeit}}\\[3mm]
        \vspace{5em}
        \textsf{\textbf{\Large{\DerTitelDerArbeit}}} \\[1.5cm]
        \vspace{5em}
        \textsf{\textbf{Studiengang Cyber Security}\\[3mm]}
        
        \vspace{2em}
    \vfill
    
    \begin{minipage}{\textwidth}
    
    \begin{tabbing}
        Wissenschaftlicher Betreuer: \hspace{1.5cm}\=\kill
        Verfasser: \> \DerAutorDerArbeit \\[1.5mm]
        Matrikelnummer: \> 4110112 \\[1.5mm]
        Kurs: \> \DieKursbezeichnung \\[1.5mm]
        Bearbeitungszeitraum: \> 06.06.2023 -- 29.08.2023\\ [1.5mm]
        Abgabedatum: \> 29.08.2023\\ [1.5mm]
        Betrieblicher Betreuer: \> Stefan Köster\\ [1.5mm]
        Gutachter der Dualen Hochschule: \> Benjamin Brindle\\
        % Ausbildungsfirma: \> \DerNameDerFirma  \\[1.5mm]
        % Betrieblicher Betreuer: \> Oliver Grimm \\
    \end{tabbing}
    
    % \vspace{2cm}
    % Unterschrift Betreuer:  \> \_\_\_\_\_\_\_\_\_\_\_\_\_\_\_\_\_\_ 
    \end{minipage}
    
    \end{center}
    
    \end{titlepage}

\pagenumbering{roman} % Römische Seitennummerierung
\normalfont

%--------------------------------
% Verzeichnisse - nicht benötige Verzeichnisse bitte auskommentieren / löschen.
%--------------------------------

% Ehrenwörtliche Erklärung ewerkl.tex einziehen
\input{ewerkl1.tex}
% \input{ewerkl2.tex}

\clearpage

\chapter*{Gender-Hinweis}

In der vorliegenden Arbeit wird darauf verzichtet, bei Personenbezeichnungen sowohl die männliche als auch die weibliche Form zu nennen. Die männliche Form gilt in allen Fällen, in denen dies nicht explizit ausgeschlossen wird, für beide Geschlechter.

%   Sperrvermerk
% \input{nondisclosurenotice}

\input{content/c1.tex}

%	Inhaltsverzeichnis
\tableofcontents



%	Abbildungsverzeichnis
\listoffigures

%	Tabellenverzeichnis
%\listoftables

%	Listingsverzeichnis
%\lstlistoflistings

% 	Algorithmenverzeichnis
%\listofalgorithms

% 	Abkürzungsverzeichnis (siehe Datei acronyms.tex!)
\input{acronyms}
\ohead{Acronyms} % Neue Header-Definition


%--------------------------------
% Start des Textteils der Arbeit
%--------------------------------
\clearpage
\ihead{\chaptername~\thechapter} % Neue Header-Definition (inner header)
\ohead{\headmark} % Neue Header-Definition (outer header)
\pagenumbering{arabic}  % Arabische Seitenzahlen



\clearpage
\pagenumbering{arabic}
\setcounter{page}{1}
%	Literaturverzeichnis
\clearpage
\ihead{}

\cleardoublepage

\input{content/c2.tex}

% LTeX: language=de-DE

\chapter{Grundlagen}
Auf der Basis einer wissenschaftlichen Literaturrecherche werden im Folgenden die benötigten Grundlagen für den weiteren Verlauf dieser Arbeit aufgezeigt:

\section{Einführung in cFront} \label{cFront-exp}
Die Abteilung \textit{cGroup Solutions} der \ac{LSY} beschäftigt sich mit der Entwicklung von Anwendungen zur Unterstützung von Airlines und Flughäfen. Dazu gehört u.a. auch das Produkt \textit{cFront}. Ziel dieses Produkts ist es, eine gemeinsame Nutzung der digitalen Infrastruktur an Flughäfen für Airlines zu ermöglichen.
Es wird an Terminals verschiedener Flughäfen eingesetzt wird. So können beispielsweise die Passagier- und Flugabfertigung verschiedener Airlines über die gleiche Hardware erfolgen. \textit{cFront} bietet somit eine standardisierte Schnittstelle zu den individuellen Applikationen der verschiedenen Airlines.

\begin{figure}[h]
	\centering 
	\includegraphics[width=1\textwidth]{img/abbildungen/MicrosoftTeams-image (3).png}
	\captionsetup{format=hang}
	\caption{cFront Startportal} \label{cFront}
\end{figure}

Es agiert dabei als Interface für mehrere Anwendungen und ermöglicht es den Kunden, diese Anwendungen zu starten. Welche Anwendungen sichtbar sind, kann je nach Kunde individuell konfiguriert werden. Ein Beispiel für ein solches Interface wird in \textbf{\ref{cFront}} dargestellt.
Die verschiedenen Icons im mittleren Bereich der Abbildungen stellen die individuellen Applikationen der jeweiligen Airline dar. Bei der Abbildung handelt es sich lediglich um Beispieldaten und keine reale Konfiguration.


\section{CIA-Triade}

Die CIA-Triade gehört zu den wichtigsten Darstellungen von Sicherheitszielen innerhalb der Informationssicherheit. Sie beschreibt die drei Schutzziele \textit{Confidentiality} (Vertraulichkeit), \textit{Integrity} (Integrität) und \textit{Availability} (Verfügbarkeit). Im Folgenden werden diese kurz beschrieben:

\begin{figure}[h]
	\centering 
	\includegraphics[width=0.5
    \textwidth]{img/abbildungen/CIA-Triad.png}
	\captionsetup{format=hang}
	\caption{CIA-Triad \cite{samonas2014cia}} \label{CIA-Triad}
\end{figure}

\paragraph*{Confidentiality} 
Die Vertraulichkeit gehört zu den wichtigsten Schutzzielen in der Informationssicherheit \cite{samonas2014cia}. Das Wort \textit{Confidentiality} kommt vom lateinischen Wort \textit{confidere} und bedeutet \textit{vertrauen} \cite{pons} \cite{samonas2014cia}. Das Schutzziel der Vertraulichkeit besagt, dass Informationen und Daten so geschützt sein müssen, dass diese nur von autorisierten Personen und für autorisierte Zwecke genutzt werden können \cite{samonas2014cia}. 
Dies beinhaltet beispielsweise Einschränkungen des Zugriffs auf Informationen und Daten, um die Privatsphäre und persönliches Eigentum zu schützen \cite{samonas2014cia}. Aber auch Verschlüsselungen, eine sichere Authentifizierung und Sicherheitsprotokolle können zur Gewährleistung der Vertraulichkeit beitragen \cite{agarwal2011security}.
Aufgrund der steigenden Wichtigkeit von wirtschaftlichen Aspekten hat die Vertraulichkeit im Vergleich zu früher an Bedeutung verloren \cite{samonas2014cia}. Häufig werden Schutzziele vernachlässigt, um im Gegenzug eine erhöhte Benutzerfreundlichkeit oder Wirtschaftlichkeit zu erreichen.

\paragraph*{Integrity}  
Das Wort \textit{Integrity} leitet sich vom lateinischen Wort \textit{tangere} ab und bedeutet so viel wie \textit{berühren} \cite{pons}. Durch die Vorsilbe \textit{In-} soll eine gegenteilige Bedeutung  im Sinne von \textit{Unberührbarkeit} entstehen \cite{samonas2014cia}. 
Die Integrität soll somit garantieren, dass Daten nicht verändert werden können, ohne dass dies bemerkt wird \cite{agarwal2011security}. Schickt ein Sender $S$ beispielsweise eine Nachricht an einen Empfänger $E$, so soll die Nachricht identisch beim Empfänger $E$ ankommen, wie sie vom Sender $S$ gesendet wurde \cite{agarwal2011security}.
Umsetzungsmöglichkeiten die Integrität zu schützen beinhalten Maßnahmen wie beispielsweise das Verwenden einer Firewall, eines \ac{IDS} oder auch digitale Signaturen \cite{agarwal2011security}.

\paragraph*{Availability}
Das Wort Availability leitet sich vom lateinischen \textit{valere} ab und bedeutet so viel wie \textit{kräftig sein}. Die Verfügbarkeit bezieht sich somit auf einen zeitnahen und zuverlässigen Zugriff auf Informationen und Daten \cite{samonas2014cia}. Zuverlässig bedeutet dabei auch, dass ein Zugriff möglichst ohne Unterbrechungen und unabhängig vom Standort möglich ist \cite{agarwal2011security}.
Verfügbarkeit kann beispielsweise durch Netzwerksicherheit (z.B. Schutz vor \ac{DDoS}) oder Fehlertoleranz (z.B. durch Limitierung von Authentifizierungsversuchen) gewährleistet werden \cite{agarwal2011security}.

Ein wichtiger Aspekt der Schutzziele ist, dass diese nicht unabhängig voneinander betrachtet werden dürfen. Vielmehr handelt es sich, um ein Zusammenspiel der verschiedenen Schutzziele (siehe \textbf{\ref{CIA-Triad}}). So kann eine einzelne Sicherheitsmaßnahme beispielsweise mehrere Schutzziele schützen. Ebenfalls lassen sich weitere Schutzziele aus den drei bestehenden ableiten. Häufig werden erweiterte Schutzziele wie beispielsweise die \textit{Authenticity} (Authentizität) oder die \textit{Non-repudiation} (Nicht-Abstreitbarkeit) \cite{samonas2014cia} definiert. Diese können dabei zumeist von einem oder mehreren Schutzzielen der CIA-Triade abgeleitet werden. Die folgende Grafik zeigt beispielhaft einige erweiterte Schutzziele und deren Bezug zu der CIA-Triade:

\begin{figure}[h]
	\centering 
	\includegraphics[width=0.8\textwidth]{img/abbildungen/Schutzziele.png}
	\captionsetup{format=hang}
	\caption{Erweiterte Schutzziele \cite{samonas2014cia}} \label{Schutzziele}
\end{figure}



\section{Arten der Authentifizierung}

Die Authentifizierung dient häufig als erste Verteidigungslinie von Systemen \cite{boonkrong2012security}. Sie gilt dabei als erweitertes Schutzziel und ist eine der wichtigsten Schutzmaßnahmen für technische Systeme. Sie übernimmt die Kontrolle über die Zugänge von Systemen und überprüft, welche Identität autorisiert ist, diese zu nutzen. In der Fachliteratur wird häufig zwischen drei verschiedenen Arten der Authentifizierung unterschieden, welche umgangssprachlich auch als \textit{Faktoren} bekannt sind. Diese sollen im Folgenden beschrieben werden:

\begin{figure}[h]
	\centering 
	\includegraphics[width=0.4\textwidth]{img/abbildungen/factors.png}
	\captionsetup{format=hang}
	\caption{Faktoren der Authentifizierung}
\end{figure}

\paragraph*{Something you know:}
Die meistgenutzte Art der Authentifizierung basiert auf dem Wissen des Nutzers. Diese Methode nutzt Informationen - welche nur dem Nutzer bekannt sind - und bestätigt so seine Identität \cite{boonkrong2012security}. Das bekannteste Verfahren ist dabei die Nutzung von Passwörtern, welche nur dem Nutzer bekannt sind. Weitere Verfahren dieser Kategorie sind beispielsweise Sicherheitsfragen. Diese werden initial vom Nutzer beantwortet und im weiteren Verlauf zur Authentifizierung abgefragt.

\paragraph*{Something you have:}
Diese Art der Authentifizierung nutzt physische Objekte, um die Identität eines Nutzers zu verifizieren. Es handelt sich dabei um Objekte, die sich lediglich im Besitz des Nutzers befinden \cite{boonkrong2012security}. Mögliche Beispiele für diese Methode sind Smartcards, welche an physische Zutrittskontrollen gehalten werden müssen oder Hardware Tokens, die für die Anmeldung an Systemen genutzt werden.

\paragraph*{Something you are:}
Diese Art der Authentifizierung basiert auf der Inhärenz. Das bedeutet, dass zur Verifizierung der Identität eines Nutzers biometrische Merkmale verwendet werden \cite{boonkrong2012security}. Dazu gehören u.a. Fingerabdrücke, Gesichtserkennungen und Iris-Scans. Diese Methode hat sich besonders im Bereich der mobilen Systeme etabliert, so bietet Apple bei seinen Smartphones beispielsweise eine Authentifizierung per Fingerabdruck (\textit{Touch ID}) oder Gesichtserkennung (\textit{Face ID}) an. Aber auch Microsoft bietet eine Authentifizierung mittels biometrischer Daten an (\textit{Windows Hello} und \textit{Hello for Business}). Diese eignet sich nicht nur für mobile Endgeräte.

Ähnlich zu der CIA-Triade lassen sich auch hier weitere Faktoren ergänzen oder ableiten. Ein weiterer Faktor ist beispielsweise \textit{something you produce} \cite{boonkrong2012security}. Diese Art der Authentifizierung leitet sich teilweise von dem Faktor \textit{something you are} ab. Sie nutzt beispielsweise die Stimme des Nutzers oder seine (digitale) Unterschrift, um seine Identität zu verifizieren \cite{boonkrong2012security}. 

Die verschiedenen Arten der Authentifizierung spielen auch bei der Unterscheidung zwischen einer \ac{SFA} und einer  \ac{MFA} eine wichtige Rolle. Wird ein einzelner Faktor genutzt, so bezeichnet man dies als \ac{SFA}. Werden mehrere Faktoren genutzt, handelt es sich um eine \ac{MFA}.

\section{Passwortbasierte Authentifizierung}\label{pw-auth}

    Die am häufigsten genutzte Methode zur Authentifizierung ist die passwortbasierte Authentifizierung \cite{boonkrong2012security} \cite{chanda2016password} \cite{yildirim2019encouraging}. Diese basiert auf dem Faktor \textit{something you know}, also auf dem Wissen der Nutzer. Zumeist handelt es sich um alphanumerische Passwörter, welche aus einer Kombination von Groß- und Kleinbuchstaben, Zahlen und Sonderzeichen bestehen \cite{chanda2016password}. Die Sicherheit informationstechnischer Systeme ist somit abhängig von der Sicherheit der genutzten Passwörter \cite{boonkrong2012security}. Trotz ihrer weitreichenden Verbreitung gelten Passwörter als eins der größten Sicherheitsrisiken für Systeme, da sie viele Schwachstellen und Angriffsvektoren aufweisen \cite{yildirim2019encouraging} \cite{farke2020you}. Laut einer Studie von \textit{Verizon} basierten 2017 81 \% der Hackerangriffe auf der Kompromittierung von Passwörtern \cite{barbosa2021provable} \cite{verizon2017}. Eine weitere Studie zeigt auf, dass 2017 Phishing E-Mails die am meisten genutzte Angriffsmethode darstellte \cite{Symantec} \cite{barbosa2021provable}. Diese sind ebenfalls häufig darauf ausgelegt, an Passwörter von Nutzern zu gelangen. Eine Vielzahl von großen Unternehmen wurden bereits Opfer von der Veröffentlichung von Passwörtern, obwohl ein hoher Aufwand betrieben wird, um diese zu schützen \cite{boonkrong2012security}. Da sich die Enthüllung der Passwörter allerdings als Angriffsziel bei Angreifern etabliert hat, ist selbst ein hoher Aufwand nicht immer ausreichend, um jene zu schützen \cite{boonkrong2012security}. Der dabei entstehende Schaden ist immens, da es sich um einen hohen Geldwert, aber u.a. auch um einen Reputationsschaden handeln kann. Trotz der bekannten Schwachstellen und bereits entwickelter alternativer Ansätze, bleibt das Passwort weiterhin die dominierendste Art der Authentifizierung \cite{ives2004domino}. Dies liegt insbesondere an der Einfachheit und dem geringen Aufwand, welcher durch eine Nutzung von Passwörtern entsteht \cite{yildirim2019encouraging}.

    Passwörter können durch verschiedene Arten von Angriffen kompromittiert werden. So können Angreifer beispielsweise Zugriff auf die Datenbank erhalten, in welcher die Passwörter gespeichert werden, aber auch auf persönlicher Ebene können Passwörter erlangt werden. Dabei spielt Social Engineering eine große Rolle. Mithilfe von Shoulder Surfing können Angreifer beispielsweise versuchen, Nutzern beim Passwort eintippen zuzuschauen. Mithilfe von Dumpster Diving können aufgeschriebene Passwörter erlangt werden. Zu den häufigsten Social Engineering Angriffen gehören allerdings die bereits beschriebenen Phishing E-Mails. Auf technischer Ebene ist ebenfalls ein Einsatz von Keyloggern möglich, welche alle Tastendrücke des Nutzers speichern. Ein häufig gewähltes und sehr effektives Mittel bei schlechten Passwörtern sind Brute-Force- und Dictionary-Angriffe. Diese kompromittieren Passwörter durch das stupide Ausprobieren aller möglichen Kombinationen oder durch die Nutzung von Tabellen, welche die meistgenutzten Passwörter beinhalten \cite{chanda2016password} \cite{morii2017research}.

    Um Passwörter resistenter gegen Brute-Force-Angriffe zu gestalten, kann eine Erweiterung des Zeichenraums oder der Passwortlänge genutzt werden. So wird die mögliche Anzahl an Kombinationen des Passworts erhöht. Je mehr mögliche Kombinationen es gibt, desto schwieriger wird es Passwörter durch Erraten zu kompromittieren \cite{chanda2016password}. Wichtig ist hierbei, dass die Erweiterung der Passwortlänge deutlich effektiver ist als die Erweiterung des Zeichenraums. Betrachtet man die Anzahl aller Elemente des Zeichenraums $Z$ und die Passwortlänge $L$, so wird die Komplexität eines Passwortes durch $Z^L$ abgebildet. Während die Erweiterung der Passwortlänge ein exponentielles Wachstum aufweist, steigt bei einer Erweiterung des Zeichenraums die Steigung lediglich linear. Die Effektivität längerer Passwörter wird ebenfalls in \textbf{\ref{EntropyvsLength}} und \textbf{\ref{timetobreak}} dargestellt. 

    \textbf{\ref{EntropyvsLength}} stellt die Entropie von Passwörtern in Abhängigkeit ihrer Länge dar. Dabei werden ebenfalls verschieden große Zeichenräume betrachtet. Es wird deutlich, dass selbst eine hohe Differenz des Zeichenraumes lediglich einen geringen Einfluss auf die Entropie hat. Unabhängig vom Zeichenraum wird aber dennoch immer eine hohe Entropie durch eine größere Länge des Passwortes erreicht. 
    
    \textbf{\ref{timetobreak}} stellt die benötigte Zeit zum Brechen von Passwörtern in Abhängigkeit zu ihrer Länge dar. Bei beiden Varianten handelt es sich um eine zu kurz gewählte Länge eines Passwortes, allerdings ist der signifikante Unterschied durch die Erweiterung der Passwortlänge  deutlich erkennbar.

    \begin{figure}[h]
        \centering 
        \includegraphics[width=0.7\textwidth]{img/abbildungen/entropy-length.png}
        \captionsetup{format=hang}
        \caption{Entropie in Abhängigkeit der Passwortlänge \cite{chanda2016password}} \label{EntropyvsLength}
    \end{figure}
    
    \begin{figure}[H]
        \centering 
        \includegraphics[width=0.7\textwidth]{img/abbildungen/length-time.png}
        \captionsetup{format=hang}
        \caption{Benötigte Zeit, um ein Passwort zu brechen in Abhängigkeit der Länge \cite{chanda2016password}} \label{timetobreak}
    \end{figure}

    Insgesamt sind bei Passwörtern zumeist zwei Angriffsvektoren betroffen: die Speicherung und der Mensch. Im Folgenden soll präziser erläutert werden, was diese beiden Angriffsvektoren so verwundbar machen:

    \textbf{Speicherung:}

    Viele Angreifer versuchen Passwörter zu kompromittieren, indem sie Zugriff auf die Datenbank erhalten, in welcher die Passwörter gespeichert sind. Mithilfe der erlangten Passwörter erhoffen sie sich zumeist einen erweiterbaren Zugriff auf Systeme oder nutzen die Passwörter, um ihre Opfer zu erpressen \cite{boonkrong2012security}. Der wichtigste Faktor für den Erfolg solcher Angriffe ist die genutzte Art der Speicherung. Abhängig von der Art wie Passwörter gespeichert sind, offenbaren sich auch verschiedene Schwachstellen \cite{chanda2016password}.
    Die schlechteste, aber immer noch genutzte Art, Passwörter zu speichern, ist die Speicherung im Klartext. Die Passwörter werden also in lesbarer Form gespeichert. Haben Angreifer also Zugriff auf die Datenbank, so können sie alle gespeicherten Passwörter ohne weiteren Aufwand auslesen \cite{chanda2016password}.

    Eine bessere Variante ist die Verschlüsselung der gespeicherten Passwörter. Der größte Kritikpunkt an dieser Variante ist allerdings, dass Verschlüsselungen zurückführbar sind. Das bedeutet mit dem Besitz des benötigten Schlüssels, lassen sich alle gespeicherten Daten ebenfalls in Klartext umwandeln. Hierbei müssen Angreifer also einen weiteren Aufwand erbringen, um an den benötigten Schlüssel zu gelangen. Sind sie allerdings im Besitz dieses Schlüssels können sie ebenfalls alle gespeicherten Passwörter auslesen \cite{chanda2016password}. 

    Um die Speicherung weiter zu optimieren, darf keine Zurückführbarkeit bestehen. Dies kann mithilfe von Hashing umgesetzt werden. Sog. Hashfunktionen erhalten einen Eingabewert und bilden diesen auf einen festen Ausgabewert ab. Dieser Ausgabewert ist nicht zurückführbar auf den ursprünglichen Eingabewert. Kompromittieren Angreifer also die Datenbank, in welcher die Passwörter gespeichert sind, können diese die gespeicherten Werte nicht direkt weiterverwenden. Auch dieser Ansatz birgt allerdings Schwachstellen. So lassen sich beispielsweise sog. Rainbow-Tabellen nutzen, um Hash-Werte zurückzuführen. Diese Tabellen beinhalten häufig genutzte Passwörter sowie ihre zugehörigen Hash-Werte. So lassen sich Hash-Werte teilweise weiterhin auf ihren Eingabewert zurückführen \cite{chanda2016password}.

    Um auch diese Schwachstelle zu verhindern, wird ein sog. Salt benötigt. Dabei wird jedes Passwort, bevor es gehashed wird, mit einem individuellen, randomisierten Wert (dem Salt) konkateniert. Somit wird verhindert, dass sich der gespeicherte Hash-Wert mithilfe von Rainbow-Tabellen vergleichen lässt. Auch eine Umsetzung mit zwei Salt-Werten ist möglich \cite{chanda2016password}.

\textbf{Faktor Mensch:}

Neben den aufgezählten technischen Aspekten stellt der Mensch selbst einen der größten Angriffsvektoren für Passwörter dar \cite{ives2004domino} \cite{yildirim2019encouraging}. Eins der größten Probleme ist, dass von Menschen erstellte Passwörter keine echten Zufallswerte sind. Das liegt insbesondere daran, dass Nutzer sich ihre Passwörter merken müssen. Je komplexer ein Passwort gestaltet ist, desto schwieriger wird es für Nutzer sich dieses zu merken - insbesondere, wenn sie sich mehrere verschiedene Passwörter merken müssen. Daher beinhalten Passwörter häufig Informationen, welche einen Bezug zum Inhaber haben. Dazu gehören beispielsweise Namen, Geburtsdaten, Adressen oder andere persönliche Informationen. Auch Passwörter, welche einfache Muster beinhalten sind sehr beliebt. Dazu gehören beispielsweise \textit{qwertz}, welches die ersten Buchstaben auf der Tastatur darstellt und \textit{123456}. Solche Passwörter können sich Menschen besser einprägen, was notwendig ist, wenn Passwörter häufig genutzt werden müssen. Aus dem identischen Grund neigen Nutzer ebenfalls dazu ein Passwort für mehrere Systeme zu nutzen \cite{boonkrong2012security} \cite{chanda2016password} \cite{yildirim2019encouraging}. 

Die genannten Faktoren führen dazu, dass die Anzahl an genutzten Kombinationen für ein Passwort deutlich geringer ist als die gesamte Menge an möglichen Kombinationen \cite{boonkrong2012security}. Das macht von Menschen erstellte Passwörter deutlich anfälliger für Angriffe, da diese einfacher zu erraten sind \cite{chanda2016password}. Dies liegt häufig auch daran, dass die Motivation der Nutzer gering ist, komplexe Passwörter zu erstellen, weil sie sich der Gefahr von schwachen Passwörtern nicht bewusst sind \cite{yildirim2019encouraging}. Kontraproduktiv können in diesem Zusammenhang auch Policies und Richtlinien zur Erstellung von Passwörtern \cite{yildirim2019encouraging} wirken. Sind die Richtlinien zur Erstellung von Passwörtern zu komplex, tendieren Nutzer bewusst dazu, Muster in das Passwort einzubauen, um sich dieses zu merken. Dies führt zu einem gegenteiligen Effekt, da die Sicherheit und die Komplexität der Passwörter dadurch sinkt. Die These, dass solche Richtlinien zu einer erhöhten Sicherheit beitragen ist somit nicht zwangsläufig korrekt \cite{yildirim2019encouraging} \cite{morii2017research}.

Ein weiteres Problem stellt die bereits genannte, mehrfache Nutzung eines Passwortes für verschiedene Systeme dar. 
Aktive Internet-Nutzer verwalten im Durchschnitt 15 Passwörter pro Tag \cite{ives2004domino}. 
Um sich also das Einprägen verschiedener Passwörter zu ersparen, wählen Nutzer tendenziell lieber ein einziges Passwort. Das führt häufig zu einem Dominoeffekt im Falle einer Passwort-Kompromittierung. 
Gelangen Angreifer an ein einzelnes Passwort des Nutzers, ist es häufig möglich mit diesem auch Zugriff auf andere Systeme zu gelangen \cite{ives2004domino} \cite{morii2017research}. 


\section{Passwortlose Authentifizierung} \label{alts}

Unter dem Sammelbegriff der passwortlosen Authentifizierung werden verschiedene Verfahren zusammengefasst, welche die Nutzung von Passwörtern ersetzen sollen. Im Gegensatz zur passwortbasierten Authentifizierung steht also nicht mehr der Faktor \textit{something you know} im Vordergrund, da das Wissen des Nutzers nicht mehr die Grundlage zur Verifizierung seiner Identität darstellen soll. Die \ac{FIDO} Allianz nutzt den Begriff passwortlose Authentifizierung beispielsweise, um eine \ac{SFA} oder \ac{MFA} mit der Hilfe eines Authentifizierungsgerätes zu beschreiben \cite{farke2020you}. 

Passwortlose Verfahren werden dabei als sicherer im Vergleich zur passwortbasierten Alternative angesehen, da viele der in \textbf{\ref{pw-auth}} aufgeführten Angriffsvektoren nicht die passwortlosen Ansätze betreffen \cite{chowhan2019password} \cite{parmar2022comprehensive}. Zudem erhofft man sich eine erhöhte Benutzerfreundlichkeit durch passwortlose Verfahren - insbesondere, weil Nutzer sich keine Passwörter mehr merken müssen und so ein geringerer Aufwand besteht \cite{chowhan2019password}.

Passwortlose Verfahren haben sich zum aktuellen Zeitpunkt jedoch noch nicht flächendeckend durchgesetzt und sind nicht annähernd so weit verbreitet wie Passwörter. Dies lässt sich auf mehrere Faktoren zurückführen. Häufig genannte Gründe innerhalb der Fachliteratur sind die Umgewöhnung der Nutzer an eine neuartige Authentifizierung, welches als Hürde zur Etablierung der passwortlosen Verfahren angesehen wird \cite{chowhan2019password}. Aber auch zusätzliche entstehende Kosten durch die Integration der neuen Verfahren können eine Verbreitung ausbremsen \cite{chowhan2019password}. Ein detaillierter Einblick in die Vor- und Nachteile in Bezug auf der Benutzerfreundlichkeit wird in \textbf{\ref{Yubikey}} gegeben.

Es gibt eine Vielzahl an Möglichkeiten eine passwortlose Authentifizierung umzusetzen. Eine der am häufigsten genutzten Varianten, ist die Nutzung eines Authentifizierungsgerätes in Kombination mit \ac{FIDO}2. Diese Variante liegt im Fokus dieser Arbeit und wird in \textbf{\ref{Yubikey}} und \textbf{\ref{fido2}} genauer beschrieben. Dennoch werden im Folgenden auch andere mögliche Alternativen kurz vorgestellt:

\textbf{Magic Link:}

Bei einem Magic Link handelt es sich um eine
Authentifizierungsmöglichkeit, bei welcher Nutzer
lediglich ihren Benutzernamen oder ihre E-Mail-Adresse
zur Anmeldung angeben müssen. Anschließend erhält der Nutzer eine E-Mail mit einem dazugehörigen Link, welcher genutzt wird, um seine Identität zu verifizieren \cite{chowhan2019password} \cite{parmar2022comprehensive}.
Dieser Link beinhaltet einen Authentication Code, welcher im Hintergrund abgeglichen und validiert wird. Ist die Validierung erfolgreich wird der Nutzer authentifiziert und angemeldet. Nach der Anmeldung verliert der Authentication Code seine Gültigkeit und somit auch der Link selbst \cite{chowhan2019password}. Der Ablauf des Verfahrens wird ebenfalls vereinfacht in \textbf{\ref{magiclink}} dargestellt.
Die Sicherheit dieses Verfahrens basiert dabei auf der Annahme, dass der Mail-Server bzw. der Zugang zum Account des Nutzers ausreichend geschützt ist. Ist diese Annahme nicht gegeben, können sich auch andere Personen mit dem Link des eigentlichen Nutzers authentifizieren, ohne autorisiert zu sein \cite{chowhan2019password}. 

\paragraph*{Vorteile:} Ein Passwort bleibt zwar in den meisten Fällen für den Zugriff auf den E-Mail Zugang notwendig, die Alternative würde aber zumindest die Anzahl an benötigten Passwörtern für Nutzer reduzieren. Zudem handelt es sich bei einem Magic Link um eine sehr benutzerfreundliche und einfach verständliche Art der Authentifizierung \cite{parmar2022comprehensive}. Auch die Implementierung und die Kosten zur Instandhaltung sind verhältnismäßig gering \cite{parmar2022comprehensive}.

\paragraph*{Nachteile:} Insbesondere im Unternehmenskontext kann die Nutzung von Spamfiltern die Benutzerfreundlichkeit von Magic Links stark beeinträchtigen. So können beispielsweise die zugehörigen Mails fälschlicherweise als Spam klassifiziert werden oder eine erhöhte Wartezeit auf die E-Mail entstehen \cite{parmar2022comprehensive}. Auch in Bezug auf das Thema Sicherheit sind einige Aspekte fragwürdig. So hängt die Sicherheit des Verfahrens von der Sicherheit des Mail-Servers ab. Ist dieser nicht ausreichend geschützt, können Angreifer Zugriff auf die Mails erhalten und sich ebenfalls mit dem Link authentifizieren \cite{chowhan2019password}. Dies kann geschehen, ohne dass der Nutzer dies überhaupt bemerkt \cite{chowhan2019password}.

\begin{figure}[h]
	\centering 
	\includegraphics[width=0.7\textwidth]{img/abbildungen/magic_link.png}
	\captionsetup{format=hang}
	\caption{Beispielhafte Umsetzung eines Magic Links} \label{magiclink}
\end{figure}

\textbf{\ac{OTP}:}

Das Konzept hinter \ac{OTP}s ähnelt dem des Magic Links. Nutzer geben ihre E-Mail-Adresse oder ihre Handynummer an (diese können ebenfalls einem Benutzernamen zugewiesen sein) und erhalten eine E-Mail/SMS, welche ein \ac{OTP} enthält \cite{chowhan2019password} \cite{parmar2022comprehensive}. 
Der Nutzer übergibt das \ac{OTP}, welches daraufhin vom System abgeglichen und validiert wird. Ist die Validierung erfolgreich, wird der Nutzer authentifiziert und angemeldet. Nach der Authentifizierung verliert das \ac{OTP} seine Gültigkeit \cite{chowhan2019password}.
Häufig werden \ac{OTP}s allerdings nicht für eine oben beschrieben \ac{SFA} genutzt, sondern dienen als zusätzlicher Faktor für eine \ac{MFA} \cite{chowhan2019password}. So können beispielsweise Authenticator Apps zur Bereitstellung von \ac{OTP}s genutzt werden, um die passwortbasierte Authentifizierung sicherer zu gestalten.
Im Gegensatz zu statischen von Anwendern gewählten
Passwörtern sind \ac{OTP}s dynamisch erzeugt und haben nur
eine geringe Lebensdauer. So wird eine höhere
Sicherheit gewährleistet, da \ac{OTP}s nur schwierig durch
stupides Erraten oder Brute-Force-Attacken erbeutet
werden können \cite{chowhan2019password}.
Für die Umsetzung von \ac{OTP}s gibt es mehrere Möglichkeiten. Zwei häufig verwendete Optionen sind \ac{HOTP} und \ac{TOTP} \cite{chowhan2019password}.
\ac{HOTP}s basieren auf der technischen Spezifikation RFC 4226. Sie werden mithilfe von einem \ac{HMAC} und unabhängig von der Zeit generiert. Neue \ac{HOTP}s können Event-basiert von dem Nutzer angefordert werden \cite{chowhan2019password}.
\ac{TOTP}s basieren auf der technischen Spezifikation RFC 6238 und werden in Abhängigkeit zu der Zeit erstellt. Sie ändern sich nach einem vordefinierten Zeitintervall und sind somit sehr kurzlebig \cite{chowhan2019password}.

\paragraph*{Vorteile:} Die Nutzung von \ac{OTP}s ist sehr effektiv für eine \ac{MFA}, da die Sicherheit im Vergleich zu einer passwortbasierten \ac{SFA} signifikant erhöht werden kann \cite{chowhan2019password} \cite{parmar2022comprehensive}. Zudem handelt es sich um eine sehr benutzerfreundliche und leicht anwendbare Methode, welche sich bereits für die Nutzung von \ac{MFA} weitreichend etabliert hat \cite{parmar2022comprehensive}. Dies liegt auch an der Vielzahl an Umsetzungsmöglichkeiten von \ac{OTP}s, da diese beispielsweise via E-Mail, SMS, Authenticator App oder auch Security Key an den Nutzer übermittelt werden können \cite{chowhan2019password} \cite{parmar2022comprehensive}. 

\paragraph*{Nachteile:} Da sich diese Arbeit auf die Nutzung einer passwortlosen Authentifizierung als \ac{SFA} fokussiert, wird es hier als Nachteil eingeordnet, dass sich die Nutzung von \ac{OTP}s hauptsächlich für die Umsetzung einer \ac{MFA} anbietet. Eine Nutzung von \ac{OTP}s als \ac{SFA} wird häufig nicht unterstützt. Zudem ist je nach Implementierung wie auch bei einem Magic Link eine Abhängigkeit auf einen anderen Dienst gegeben, welche die Sicherheit des Verfahrens beeinträchtigen können.

\textbf{Biometrische Daten:}

Eine bereits weitreichend etablierte Methode zur passwortlosen Authentifizierung ist die Nutzung von biometrischen Daten. Diese wird insbesondere im Bereich der mobilen Endgeräte genutzt \cite{parmar2022comprehensive}. Dabei werden einzigartige biometrische Merkmale des Nutzers genutzt um seine Identität zu verifizieren. Dazu gehören beispielsweise Fingerabdrücke oder eine Gesichtserkennung \cite{parmar2022comprehensive}. Im Bereich der Smartphones wird dieser Ansatz u.a. von Apple durch die Technologien \textit{Touch ID} und \textit{Face ID} integriert. Aber auch Microsoft ermöglicht mittlerweile eine Authentifizierung mittels biometrischer Daten an (\textit{Windows Hello} und \textit{Hello for Business}). Diese Technologie lässt sich ebenfalls für eine Integration in den Unternehmenskontext nutzen und ist nicht nur für mobile Endgeräte verfügbar. Wichtig ist hierbei, dass die Authentifizierung selbst auf der Nutzung von Schlüsselpaaren basiert. Lediglich der Zugriff auf die Schlüssel wird mithilfe der biometrischen Daten geschützt.

\paragraph*{Vorteile:} Viele mobile Endgeräte arbeiten bereits mit biometrischen Daten. Daher ist für Nutzer keine große Umgewöhnung an die neue Art der Authentifizierung notwendig \cite{parmar2022comprehensive}. Da biometrische Daten nahezu einzigartig sind, sind diese ebenfalls deutlich schwieriger anzugreifen als Passwörter. Durch die Unterstützung von Microsoft mit Hello for Business ist diese Methode bereits für den Unternehmenskontext verfügbar \cite{parmar2022comprehensive}.

\paragraph*{Nachteile:} Äußere Bedingungen können die Erkennung von biometrischen Daten beeinträchtigen. So kann beispielsweise schlechtes Licht bei einer Gesichtserkennung oder staubige Umgebungen bei einem Fingerabdruckscanner die Erkennung beeinträchtigen \cite{parmar2022comprehensive}. Zudem können sich biometrische Daten im Laufe der Zeit verändern. Auch Verletzungen oder Krankheiten können zu Veränderungen der biometrischen Daten beitragen \cite{boonkrong2012security}. Auch wenn Microsoft bereits biometrische Daten unterstützt, ist es im Unternehmenskontext häufig so, dass verschiedene Hersteller und Geräte genutzt werden. Diese unterstützen nicht immer zwangsweise biometrischen Daten oder sind nicht untereinander kompatibel \cite{parmar2022comprehensive}.

\textbf{Öffentliche/Private Schlüsselpaare:}

Bei der Nutzung von öffentlichen und privaten Schlüsselpaaren handelt es sich um eine asymmetrische Verschlüsselung. Dabei werden ein öffentlicher und ein privater Schlüssel generiert. Der öffentliche Schlüssel wird an den Server übermittelt und der private Schlüssel wird auf dem Gerät des Nutzers gespeichert. Die Authentifizierung erfolgt durch die Nutzung des privaten Schlüssels. Mithilfe des öffentlichen Schlüssels kann der Server die Identität des Nutzers verifizieren. Eine genaue Beschreibung des Verfahrens anhand des \ac{FIDO}2 Protokolls wird in \textbf{\ref{fido2}} aufgeführt.


\section{YubiKey} \label{Yubikey}
Ein Security Key ist eine Hardware, welche es ermöglicht einen Nutzer zu authentifizieren. Der Nutzer muss dazu mit dem Security Key interagieren (beispielsweise durch einen Knopfdruck) \cite{reynolds2018tale}. In der Fachliteratur lassen sich viele verschiedene Bezeichnungen für Security Keys finden. Dazu gehören beispielsweise \textit{Security Token}, \textit{Hardware Token}, \textit{Authentifizierungsgerät}. Um eine einheitliche Bezeichnung zu gewährleisten, wird in dieser Arbeit der Begriff \textit{Security Key} genutzt. Beispielhaft wird in dieser Arbeit ein YubiKey der Series 5 (siehe \textbf{\ref{yubikey}}) als Referenzmodell genutzt. Dies basiert auf der Entscheidung, welche in \textbf{\ref{secwahl}} getroffen wird. Grundsätzlich sind die Funktionsweisen der verschiedenen Security Keys allerdings sehr ähnlich.

Der YubiKey 5 ermöglicht grundsätzlich drei Arten der Authentifizierung:

\begin{enumerate}
    \item Eine \ac{SFA}, welche Passwörter durch ein passwortloses \textit{tap-n-go} Verfahren ersetzt \cite{yuibkey2023fido2}.
    \item Eine Nutzung des Security Keys als zusätzlicher Faktor für eine \ac{2FA}. Somit wird das Passwort zusätzlich abgesichert. Der Security entspricht somit dem zweiten Faktor (\textit{something you have}) \cite{yuibkey2023fido2}.
    \item Eine passwortlose \ac{MFA} mithilfe einer zusätzlichen PIN für den Security Key \cite{yuibkey2023fido2}.
\end{enumerate}

\begin{figure}[h]
	\centering 
	\includegraphics[width=0.3\textwidth]{img/abbildungen/yubikey.jpeg}
	\captionsetup{format=hang}
	\caption{YubiKey der Series 5 \cite{yubikeypic}} \label{yubikey}
\end{figure}

\subsection{Usability}
Es gibt bereits Fachliteratur, welche sich mit der Benutzerfreundlichkeit von Security Keys beschäftigen. Diese beziehen sich zumeist auf die Implementierung von Security Keys in Kleinunternehmen. Die gesammelte Recherche soll genutzt werden um im Folgenden Vor- und Nachteile der Nutzung von Security Keys in Bezug auf deren Benutzerfreundlichkeit zu erläutern. Dies wird in \textbf{\ref{questions}} als Basis für die Evaluation eines Fragebogens für die \ac{LSY} genutzt.

\textbf{Vorteile:}

\begin{itemize}
    \item Ergebnisse zeigen, dass Nutzer grundsätzlich bereit sind, Passwörter durch passwortlose Verfahren zu ersetzen \cite{lyastani2020fido2}.
    \item Passwortlose Verfahren mit einem Security Key weisen eine höhere Akzeptanz auf als traditionelle passwortbasierte Verfahren \cite{lyastani2020fido2}.
    \item Es handelt sich um eine implizite Garantie, dass sich lediglich Nutzer authentifizieren können, welche auch im Besitz des Security Keys sind \cite{lyastani2020fido2}.
    \item Durch die Nutzung eines Security Keys und \ac{FIDO}2 kann die Benutzerfreundlichkeit erhöht werden, da Nutzer sich keine Passwörter mehr merken müssen. Häufig wird das Verwalten der immer höher werdenden Anzahl an Passwörtern als Problem angesehen \cite{farke2020you} \cite{lyastani2020fido2}.
    \item Es wird ein deutlich geringerer kognitiver Aufwand benötigt, da Nutzer keine neuen Passwörter mehr erstellen müssen \cite{lyastani2020fido2}.
    \item Zum aktuellen Zeitpunkt wird \ac{FIDO}2 bereits von einer Vielzahl an Browsern unterstützt (und somit auch die Nutzung von Security Keys). Zusätzlich bieten immer mehr Online-Dienste die Möglichkeit an, sich mithilfe von \ac{FIDO}2 zu authentifizieren \cite{farke2020you} \cite{lyastani2020fido2}.
    \item Es handelt sich um offene und standardisierte Protokolle. Das verhindert verschieden Lösungsansätze verschiedener Hersteller und führt zu einer unabhängigeren und universellen Lösung \cite{farke2020you}.
\end{itemize}

\textbf{Nachteile:}

\begin{itemize}
    \item Im Falle einer \ac{SFA} wird der Verlust des Security Keys als größtes Problem angesehen. Bei einem Verlust verliert der Nutzer seinen Zugriff. Aktuell gibt es noch keine sichere und effiziente Möglichkeiten, um den Zugriff wiederherzustellen \cite{lyastani2020fido2}.
    \item Da es sich um zusätzliche Hardware handelt, kann diese ebenfalls kaputtgehen \cite{farke2020you}.
    \item Im Unternehmenskontext, kann die Verwaltung und Verteilung der Security Keys zu einem Problem werden \cite{farke2020you}. 
    \item Da es sich um Hardware handelt, können Zugänge nicht an vertraute Personen weitergegeben werden, da der Zugang nur mit dem Security Key möglich ist \cite{lyastani2020fido2}.
    \item Ohne den Security Key sind keine spontanen Logins möglich \cite{lyastani2020fido2}.
    \item Bereits der Aufwand den Security Key aus der Tasche zu holen, ist für manche Nutzer eine Hürde \cite{farke2020you}.
    \item Es wird ein physischer Aufwand benötigt, da der Security Key mitgeführt werden muss \cite{lyastani2020fido2}.
    \item Security Keys sind häufig mit Kosten verbunden, welche vom Nutzer getragen werden müssen \cite{lyastani2020fido2}.
    \item Nutzern fällt es schwer neues Verfahren für die Authentifizierung zu nutzen, da sie sich an das alte Verfahren gewöhnt haben. Das führt dazu, dass Nutzer das neue Verfahren als kompliziert und ungewohnt empfinden. Sie verfügen häufig nicht über das nötige Wissen, um die Funktion und Sicherheit des Verfahrens zu verstehen \cite{lyastani2020fido2}.
    \item Selbst Nutzern, welchen das Konzept der passwortlosen Authentifizierung gefällt, nutzen in der Praxis häufig weiterhin Passwörter \cite{farke2020you}.
    \item Nutzer wollen keine Angewohnheiten verändern, wenn sie nicht dazu gezwungen sind \cite{farke2020you}.
    \item Nutzer verwenden lieber Passwörter, da sie das Konzept und die Technologie besser verstehen \cite{lyastani2020fido2}.
    \item Eine Authentifizierung mit einem Security Key ist nicht zwangsweise schneller als die Nutzung von Passwortmanagern \cite{farke2020you}.
    \item Allgemein fällt das Feedback von Nutzern weniger positiv aus, wenn diese vorher bereits Passwortmanager genutzt haben \cite{farke2020you}.
\end{itemize}

Insgesamt lassen sich noch nicht alle Szenarien mit Security Keys abdecken. Es gibt noch spezielle Fälle, in welchen die Nutzung von Passwörtern weiterhin notwendig ist \cite{lyastani2020fido2}. Auffällig jedoch ist, dass die Teilnehmer der Studien häufig einen skeptischen Blick auf die neuartige Authentifizierung werfen. Trotz der gesammelten Vor- und Nachteile kann keine Annahme darüber getroffen werden, ob die Vor- oder die Nachteile überwiegen und ob die Nutzung von Security Keys grundsätzlich gegenüber der passwortbasierten Alternative bevorzugt wird. Dies wird in \textbf{\ref{questions}} mithilfe der hier erarbeiteten Grundlage genauer analysiert.

\section{FIDO2} \label{fido2}

Es gibt bereits viele Alternativen zur passwortbasierten Authentifizierung. Einzelne wurden in \textbf{\ref{alts}} bereits beispielhaft beschrieben. Der Großteil dieser Alternativen wird allerdings nur in einem sehr geringen Ausmaß genutzt \cite{farke2020you}. \ac{FIDO}2 gehört zu den passwortlosen Verfahren, welche am meisten unterstützt werden. Das Protokoll wird von der \ac{FIDO} Allianz und dem \ac{W3C} entwickelt und bereitgestellt. Die \ac{FIDO} Allianz ist eine Organisation mit weltweit über 250 Mitgliedern. Darunter befinden sich Unternehmen wie Google, Microsoft, Apple, Amazon, Visa und viele mehr  \cite{farke2020you} \cite{lyastani2020fido2}. Das \ac{W3C} wurde 1994 - mit dem Ziel das Wachstum des Internets zu gewährleisten - von Tim Berners-Lee gegründet. Es hat über 300 Mitglieder, darunter auch die hier aufgezählten Mitglieder der \ac{FIDO} Allianz \cite{w3cabout}. Aus diesem Grund wird \ac{FIDO}2 von vielen Browsern standardmäßig unterstützt. Zudem gibt es eine Vielzahl an \ac{FIDO}2 kompatiblen Authentifizierungsgeräte, darunter u.a. auch Security Keys und Smartphones \cite{lyastani2020fido2}.

Das Ziel bei der Entwicklung von \ac{FIDO}2 ist es Nutzer zu authentifizieren, ohne dass diese ein Passwort verwenden müssen. Statt einer Nutzung von Passwörtern basiert das Protokoll auf der Nutzung von Schlüsselpaaren. In Kombination mit \ac{FIDO}2 können Security Keys als Authentifizierungsgerät genutzt werden \cite{barbosa2021provable} \cite{morii2017research}. Diese können zusätzlich ebenfalls mit einer PIN oder biometrischen Daten abgesichert werden. Dabei ist die PIN nicht mit einem Passwort gleichzusetzen. Die PIN wird lediglich für das Authentifizierungsgerät genutzt und wird auch nur auf diesem gespeichert \cite{farke2020you} \cite{barbosa2021provable}. \ac{FIDO}2 unterstützt somit sowohl \ac{SFA} als auch \ac{MFA} \cite{farke2020you} \cite{lyastani2020fido2}. Zudem wird eine hohe Sicherheit ermöglicht, da Zugangsdaten bereitgestellt werden, welche nicht von Phishing oder Datenlecks betroffen sein können \cite{lyastani2020fido2}. Dies ist der Fall, da keine geteilten Geheimnisse zwischen Nutzer und Dienst existieren, welche auf einem Server gespeichert werden \cite{morii2017research}. 

\begin{figure}[h]
	\centering 
	\includegraphics[width=1\textwidth]{img/abbildungen/fido2_usability.png}
	\captionsetup{format=hang}
	\caption{Mögliche Vorteile von FIDO2 \cite{lyastani2020fido2}} \label{fido2-pros}
\end{figure}

In \textbf{\ref{fido2-pros}} werden die möglichen Vor- und Nachteile von \ac{FIDO}2 aufgezeigt. Dabei steht die Zeile \textit{1FA} für eine \ac{SFA} mithilfe von \ac{FIDO}2, welche mit der Nutzung von Passwörtern verglichen wird. Orange markiert in der \textit{1FA}-Zeile sind Bereiche, welche grundsätzlich von \ac{FIDO}2 abhängig sind. Weiß hinterlegte Bereiche sind zusätzlich abhängig vom genutzten Authentifizierungsgerät. Auffällig ist, dass insbesondere der Bereich \textit{Sicherheit} viele Vorteile bietet, welche grundsätzlich durch \ac{FIDO}2 ermöglicht werden und unabhängig vom genutzten Authentifizierungsgerät sind. Einige dieser Vorteile wurden bereits im Rahmen dieser Arbeit betrachtet. So lassen sich \ac{FIDO}2 Nutzerdaten beispielsweise nicht erraten oder phishen. Ebenfalls können diese nicht von Datenlecks betroffen sein und auch nicht durch Shoulder Surfing erlangt werden. Ein aufgezeigter Nachteil im Vergleich zu Passwörtern ist jedoch die Möglichkeit des Diebstahls eines Authentifizierungsgerätes. Dies ist allerdings abhängig vom genutzten Authentifizierungsgerät und ergibt sich nicht aus der Nutzung von \ac{FIDO}2. Auch auffällig ist die große Abhängigkeit zum Authentifizierungsgerät im Bereich der Benutzerfreundlichkeit. Dort sind grundsätzlich zwar Vorteile gegenüber der Nutzung von Passwörtern möglich, aber nicht automatisch durch die Nutzung von \ac{FIDO}2 gegeben. 

\begin{figure}[h]
	\centering 
	\includegraphics[width=0.8\textwidth]{img/abbildungen/fido2_com.png}
	\captionsetup{format=hang}
	\caption{FIDO2 Teilnehmer \cite{barbosa2021provable}} \label{fido2-channels}
\end{figure}

\begin{figure}[h]
	\centering 
	\includegraphics[width=0.8\textwidth]{img/abbildungen/fido2_flow.png}
	\captionsetup{format=hang}
	\caption{Vereinfachter FIDO2 Ablauf \cite{barbosa2021provable}} \label{fido2-simple}
\end{figure}

Bei der Nutzung von \ac{FIDO}2 werden vier Kommunikationskanäle genutzt. Diese werden in \textbf{\ref{fido2-channels}} abgebildet. Es besteht eine Kommunikation zwischen User, Authentifizierungsgerät und Client. Der Client ist üblicherweise ein Browser und übernimmt zusätzlich die Kommunikation mit dem Server. Vereinfacht wird der \ac{FIDO}2 Ablauf in \textbf{\ref{fido2-simple}} dargestellt. Der User meldet sich über den Client beim Server an (oder registriert sich erstmalig). Daraufhin sendet der Server eine Challenge an den Client, welche vom Authentifizierungsgerät signiert werden muss. Der Client gibt die Challenge an das Authentifizierungsgerät weiter. Bevor dieser die Challenge signieren kann, muss der User die Aktion mithilfe einer Geste am Authentifizierungsgerät autorisieren (z.B. einem Knopfdruck). Anschließend signiert das Authentifizierungsgerät die Challenge und sendet diese an den Client zurück. Der Client übergibt die signierte Challenge an den Server. Der Server kann die Signatur mithilfe des öffentlichen Schlüssels verifizieren und den User authentifizieren \cite{farke2020you} \cite{lyastani2020fido2}.

Dieser aufgezeigte Ablauf besteht aus den zwei Subprotokollen \ac{CTAP2}, welches für die Kommunikation zwischen Client und Authentifizierungsgerät genutzt wird und WebAuthn, welches für die Kommunikation zwischen Client und Server zuständig ist. Dabei wird WebAuthn von dem \ac{W3C} spezifiziert und \ac{CTAP2} von der \ac{FIDO} Allianz \cite{farke2020you}. Im weiteren Verlauf soll der genauere Ablauf der beiden Subprotokolle detaillierter betrachtet werden.

\subsection{WebAuthn}
Bei WebAuthn handelt es sich um einen Standard, welcher es Webanwendungen ermöglicht, Nutzer zu authentifizieren \cite{lyastani2020fido2}. Seit 2019 gehört WebAuthn zu den offiziellen Webstandards \cite{farke2020you}. Er spezifiziert ein standardisiertes JavaScript \ac{API} zur Authentifizierung von Nutzern. Ein großer Vorteil von WebAuthn ist, dass es Webanwendungen ermöglicht wird eine Authentifizierung zu integrieren, welche resistent gegenüber Phishing, Datenlecks und Passwortdiebstahl ist. Anstelle von geteilten Geheimnissen nutzt WebAuthn public-key Kryptografie, um einzigartige Zugangsdaten für jede Webanwendung zu erstellen. Diese werden nur auf dem Authentifizierungsgerät des Nutzers gespeichert \cite{farke2020you}. Es handelt sich dabei um ein sog. Challenge-Response-Verfahren zwischen einem Client und einem Server \cite{barbosa2021provable}. Die Sicherheit des Standards basiert auf dem Beweis, dass RSASSA PKCS1-v1\_5 und RSASSA-PSS als \ac{EUF-CMA} gelten und der Annahme, dass SHA-256 kollisionsresistent ist \cite{barbosa2021provable}.

WebAuthn unterstützt zwei Operationen: Registrierung und Anmeldung. In der Registrierungsphase sendet der Server dem Authentifizierungsgerät über den Client eine zufällige Challenge. In dieser Phase signiert das Authentifizierungsgerät mithilfe seines privaten Schlüssels die Challenge und sendet zusätzlich öffentliche Anmeldedaten für zukünftige Anmeldungen an den Server. Meldet sich ein bereits registrierter Nutzer an, wird die Challenge des Servers ebenfalls von dem Authentifizierungsgerät signiert zurück an den Server gesendet. Der Server kann die Signatur mithilfe des öffentlichen Schlüssels verifizieren und den Nutzer authentifizieren \cite{barbosa2021provable}.

Dieser Prozess wird in \textbf{\ref{fido2-process}} genauer beschrieben (schwarz dargestellt). Der Server \textit{S} sendet eine challenge message $m_{rch}$ über den Client $C$ an das Authentifizierungsgerät. Diese Challenge beinhaltet eine randomisierte Nonce, Parameter $UV$(beispielsweise, ob eine Nutzerverifizierung notwendig ist) und optional einen wert \textit{tb}, welcher den zugrunde liegenden Kanal eindeutig identifiziert (typischerweise eine \ac{TLS} Verbindung). Der Client $C$ erhält die challenge message $m_{rch}$ und wandelt diese in eine command message $m_{rcom}$ und eine client mesasage $m_{rcl}$ um. Die command message $m_{rcom}$ wird an das Authentifizierungsgerät $T$ übermittelt. Das Authentifizierungsgerät $T$ erzeugt ein Schlüsselpaar, welches an den Server $S$ gebunden ist und diesem ermöglicht, eine Verifizierung während der folgenden Authentifizierungsphase durchzuführen. Die erzeugten Daten werden dabei nur auf dem Authentifizierungsgerät $T$ gespeichert. Zudem gibt das Authentifizierungsgerät $T$ eine response message $m_{rrsp}$ aus. Der Client übergibt diese und die client message $m_{rcl}$ an den Server $S$. Die response message $m_{rrsp}$ beinhaltet einen \textit{attestation type}, welcher es dem Server $S$ ermöglicht eine Verifizierung während der Registrierungsphase durchzuführen und beinhaltet den öffentlichen Schlüssel. WebAuthn 2 unterstützt fünf \textit{attestation types}. Häufig werden die types \textit{None} und \textit{Basic} verwendet. Die restlichen types sind \textit{Self}, \textit{AttCA} und \textit{AnonCA}. \cite{bindel2022fido2}

Ist ein Nutzer bereits registriert, so empfängt der Client eine challenge message $m_{ach}$ von Server $S$ und wandelt diese in eine command message $m_{acom}$ und eine client message $m_{acl}$ um. Die command message $m_{acom}$ wird an das Authentifizierungsgerät $T$ übermittelt. Das Authentifizierungsgerät $T$ erzeugt eine response message $m_{arsp}$, welche mit dem privaten Schlüssel signiert wird und sendet diese an den Server $S$ (über den Client $C$). Der Server $S$ akzeptiert die response message $m_{arsp}$ und die client message $m_{acl}$ nur, wenn sie sich mit dem dazugehörigen öffentlichen Schlüssel und privaten Schlüssel des Servers $S$ verifizieren lassen. \cite{bindel2022fido2}

\begin{figure}[H]
	\centering 
	\includegraphics[width=1\textwidth]{img/abbildungen/Fido2.png}
	\captionsetup{format=hang}
	\caption{FIDO2 Darstellung \cite{bindel2022fido2}} \label{fido2-process}
\end{figure}

\subsection{CTAP2}

\ac{CTAP2} wurde 2018 als internationaler Standard der \ac{ITU-T} anerkannt \cite{barbosa2021provable}. Es handelt sich um ein Protokoll auf der Anwendungsebene, welches für die Kommunikation zwischen eines WebAuthn Clients und eines konformen Authentifizierungsgerätes genutzt wird. Das Authentifizierungsgerät kann dabei ein externes Gerät sein, z.B. ein Security Key, welcher über USB, Bluetooth oder NFC eine Verbindung mit dem Client aufbaut. Aber auch interne Hardware eines Gerätes wie beispielsweise ein Trusted Platform Module können als Authentifizierungsgerät genutzt werden \cite{lyastani2020fido2}.

Mithilfe von \ac{CTAP2} wird die Kommunikation zwischen einem Authentifizierungsgerät und einem Client spezifiziert. Der Client ist dabei üblicherweise ein Webbrowser. Das Ziel ist es zu garantieren, dass der Client das Authentifizierungsgerät nur nutzen darf, wenn der Nutzer dies autorisiert. Dafür muss der Nutzer beispielsweise einen Knopf am Authentifizierungsgerät drücken und/oder sich mithilfe einer PIN oder eines biometrischen Merkmals beim Authentifizierungsgerät authentifizieren. Das Ziel von \ac{CTAP2} ist es zunächst den Client an das Authentifizierungsgerät zu binden. Ist dies nicht der Fall, so wird es dem Client nicht ermöglicht sich zu authentifizieren. Dies wird auch in \textbf{\ref{fido2-process}} abgebildet. Die blauen Bereiche stellen den Prozess von \ac{CTAP2} dar. Bevor die command message $m_{rcom}$ an das Authentifizierungsgerät $T$ gelangt, muss der Nutzer mithilfe der persönlichen PIN $t$ die Aktion über den  Client $C$ autorisieren. Nach der Eingabe erfolgt die Übergabe an das Authentifizierungsgerät $T$, welcher diese validiert. Bei nicht erfolgreicher Validierung wird der Prozess abgebrochen \cite{barbosa2021provable}.

\ac{CTAP2} besteht dabei aus vier Phasen. Diese werden im Folgenden kurz erläutert. Da eine zu detaillierte Beschreibung den Rahmen dieser Arbeit übertreffen würde, werden die Phasen lediglich in vereinfachter Form erläutert. Eine detaillierte Darstellung ist \textbf{\ref{ctap2-process}} und wird in \cite{barbosa2021provable} genauer beschrieben. Folgende Phasen beinhaltet \ac{CTAP2}:

\begin{enumerate}
    \item In der Reboot Phase wird mithilfe eines unauthentifizierten \ac{ECDH} nach der NIST P-256 Spezifikation ein Schlüsselpaar $(a, aG)$ generiert (in \textbf{\ref{ctap2-process}} bezeichnet als ECKG). Zudem wird ein pinToken $pt$ erstellt und der mismatch counter $m$ auf den Wert drei gesetzt. Der mismatch counter $m$ wird genutzt, um die Anzahl der fehlgeschlagenen PIN Eingaben zu zählen \cite{barbosa2021provable}.  
    \item In der Setup Phase erfolgt ein unauthentifizierter \ac{ECDH} Schlüsselaustausch, gefolgt von der Übertragung der verschlüsselten PIN $c_p$ des Nutzers an das Authentifizierungsgerät $T$. Der geteilte Schlüssel $K$ für die Entschlüsselung ermittelt sich durch das Hashen der x-Koordinate $abG.x$ des \ac{ECDH}. Durch die Nutzung von AES-256-CBC $CBC_0$ und des Schlüssels $K$ lässt sich die PIN $c_p$ entschlüsseln. Diese wird vom Authentifizierungsgerät $T$ validiert. Ist die Validierung erfolgreich, wird der Hash des PIN $pin_u$ als statisches Geheimnis $st_T.s$ lokal auf dem Authentifizierungsgerät $T$ gespeichert und der retries counter $st_T.n$ auf den Standardwert von acht gesetzt. Schlagen diese acht Versuche fehl, so wird das Authentifizierungsgerät $T$ gesperrt \cite{barbosa2021provable} \cite{bindel2022fido2}.
    \item Die Binding Phase beginnt ebenfalls mit einem unauthentifizierten \ac{ECDH}. Darauf folgt eine Übertragung der erst gehashten und dann verschlüsselten PIN $c_{ph}$ vom Client $C$ an das Authentifizierungsgerät $T$. Diese wird vom Authentifizierungsgerät $T$ entschlüsselt und mit dem statischen Geheimnis $st_T.s$ verglichen. Ist dieser Vergleich erfolgreich, wird der pinToken $pt$ sowohl beim Authentifizierungsgerät $T$ als auch beim Client $C$ als binding state gesetzt \cite{barbosa2021provable} \cite{bindel2022fido2}.
    \item Zuletzt findet die Validierungs- und Autorisierungsphase statt. Der Client $C$ schickt einen Befehl mit einem HMAC Tag an das Authentifizierungsgerät $T$. Dieses validiert den Tag und autorisiert den Befehl nur, wenn eine \textit{positive decision} $d$ des Nutzers vorliegt (beispielsweise einem Knopfdruck) \cite{barbosa2021provable} \cite{bindel2022fido2}.
\end{enumerate}

Eine in der Fachliteratur häufig genannte Kritik an \ac{CTAP2} ist die Verwendung des unauthentifizierten \ac{ECDH}, welcher während der Binding und Setup Phase genutzt wird. Dieser kann von \ac{MITM} Angriffen betroffen sein \cite{barbosa2021provable}.
Zudem wird auch die Umsetzung der pinTokens kritisiert. Jedem Authentifizierungsgerät wird beim Hochfahren je ein pinToken zugeteilt. Das bedeutet, dass mehrere Authentifizierungsgeräte den gleichen pinToken besitzen können. Dadurch wird die Sicherheit von \ac{CTAP2} limitiert \cite{barbosa2021provable}.

\begin{figure}[H]
	\centering 
	\includegraphics[width=1\textwidth]{img/abbildungen/ctap2-process.png}
	\captionsetup{format=hang}
	\caption{CTAP2 \cite{bindel2022fido2}} \label{ctap2-process}
\end{figure}

\subsubsection*{CTAP2.1}
Um diese Kritikpunkte zu beheben wurde CTAP2.1 entwickelt. CTAP2.1 basiert nicht mehr auf unauthentifizierten \ac{ECDH}, sondern auf einem sog. \ac{puvProtocol}. In Verbindung mit WebAuthn gilt CTAP2.1 als \ac{PQ} bereit, da ein Operationsmodus ermöglicht wird, welcher nur kryptografische Primitive, digitale Signaturen und \ac{KEM} verwendet \cite{bindel2022fido2}.

Zusätzlich werden die pinTokens im Vergleich zu \ac{CTAP2} anders erstellt. Während der pinToken bei \ac{CTAP2} aus mehreren 128 Bit-Blöcken besteht und keine maximale Begrenzung der Länge besitzt, wird der pinToken bei CTAP2.1 als pinUvAuthToken definiert. Dieser hat eine feste Länge von 128 oder 256 Bit. Der pinToken von \ac{CTAP2} wird bis zum nächsten Neustart wiederverwendet. Der pinUvAuthToken von CTAP2.1 wird nach jeder erfolgreichen Authentifizierung neu generiert. So besteht ein verringertes Risiko, dass sich mehrere Authentifizierungsgeräte den gleichen pinToken teilen. Das führt dazu, dass CTAP2.1 eine \ac{SUF}-t' Sicherheit aufweist und \ac{CTAP2} lediglich eine \ac{UF}-t' Sicherheit \cite{bindel2022fido2}. Ebenfalls ermöglicht CTAP2.1 die Nutzung von biometrischen Merkmalen, anstelle von einer individuellen PIN \cite{bindel2022fido2}.


\subsection{Sicherheit}

\ac{FIDO}2 wird in der Fachliteratur als grundsätzlich sicheres Protokoll angesehen. Es ist eine Erweiterung des \ac{FIDO} \ac{U2F} Protokolls und beinhaltet geprüfte asymmetrische Kryptografie \cite{lyastani2020fido2} \cite{farke2020you}. Im Folgenden sollen kurz die wichtigsten Vorteile im Aspekt der Sicherheit aufgezeigt werden:


\begin{itemize}
    \item Es gibt keine geteilten Geheimnisse zwischen Usern und Diensten, welche durch Phishing oder Datenlecks kompromittiert werden können \cite{farke2020you} \cite{lyastani2020fido2}. 
    \item Dasselbe Authentifizierungsgerät ist für mehrere Dienste nutzbar, ohne dass sich dabei eine Verknüpfung zurückführen lässt \cite{lyastani2020fido2} \cite{farke2020you}.
    \item Es kann lediglich die Sitzung kompromittiert werden und nicht die Zugangsdaten selbst \cite{morii2017research}.
    \item Authentifizierungsgeräte lassen sich mit zusätzlichen PINs oder biometrischen Merkmalen absichern, um sich ebenfalls vor Diebstahl schützen \cite{barbosa2021provable}.
    \item Zugangsdaten können nicht durch systematisches erraten kompromittiert werden \cite{barbosa2021provable}.
    \item Wird ein Authentifizierungsgerät gestohlen, kann dieser nur genutzt werden, wenn ebenfalls der PIN bekannt ist \cite{barbosa2021provable}.
\end{itemize}

Die Arbeit \cite{barbosa2021provable} beschreibt die Sicherheit mit folgendem Szenario:

\begin{enumerate}
    \item Der Nutzer besitzt ein Authentifizierungsgerät, welcher mit einem Knopf oder ähnlichen ausgestattet ist.
    \item Das Authentifizierungsgerät ist mit einer geheimen PIN oder biometrischen Merkmalen geschützt.
    \item Der Nutzer autorisiert vertrauten Clients auf das Authentifizierungsgerät zuzugreifen.
    \item Der Nutzer verbindet sein Authentifizierungsgerät mit mehreren Clients und nutzt diese um sich bei mehreren Webdiensten zu registrieren/anzumelden.
\end{enumerate}

In diesem Fall ist versichert, dass:

\begin{enumerate}
    \item Die Authentifizierung von dem Authentifizierungsgerät durchgeführt wurde, welcher die genutzten Zugangsdaten bei dem Webdienst registriert hat.
    \item Ein autorisierter Befehl auf das Authentifizierungsgerät zugegriffen hat.
    \item Dieser autorisierte Befehl von einem autorisierten Client beauftragt wurde (sollte der Nutzer die korrekte PIN eingegeben haben).
\end{enumerate}

Unter der Voraussetzung, dass:

\begin{enumerate}
    \item Das Authentifizierungsgerät nicht gestohlen wurde.
    \item Die PIN des Authentifizierungsgerätes nicht kompromittiert wurde.
    \item Der autorisierte Client nicht kompromittiert wurde (erfordert eine korrekte Ausführung von \ac{CTAP2} und, dass der Client ist nicht von böswilliger Software betroffen ist).
\end{enumerate}



% LTeX: language=de-DE

\chapter{Umsetzung}
Im Folgenden wird die Umsetzung einer Integration eines Security Keys in Kombination mit \ac{FIDO}2 innerhalb der \ac{LSY} beschrieben. Dabei wird der aktuelle Stand berücksichtigt und begründet ein passender Ansatz für die Integration gewählt. Zusätzlich wird die Umsetzung zum Aspekt der Benutzerfreundlichkeit analysiert.

\section{Aktueller Stand der LSY} \label{current}
Innerhalb der \ac{LSY} werden aktuell verschiedene Verfahren zur Authentifizierung genutzt. Grundsätzlich besteht eine zentrale Nutzerverwaltung innerhalb der \ac{LSY}, welche innerhalb einer Microsoft Azure \ac{AD} verwaltet wird. Wie diese an eine Applikation gebunden ist, ist dabei nicht vorgegeben und kann von den jeweiligen Abteilungen individuell festgelegt werden. Auch Lösungsansätze ohne die Nutzung der Azure \ac{AD} sind je nach Abteilung möglich. Die einzige zentrale Schnittstelle ist somit die genannte Azure \ac{AD}.

Diese bestehende Nutzerverwaltung basiert auf einer passwortbasierten Authentifizierung inklusive \ac{MFA}. Das Passwort muss von allen Nutzern in regelmäßigen Abständen (drei Monaten) geändert werden. Der weitere Faktor für die Nutzung der \ac{MFA} kann frei von den Nutzern gewählt werden, solange er mit der Azure \ac{AD} kompatibel ist. Zusätzlich zu regelmäßigen Passwortänderungen, gibt es Richtlinien zur Passworterstellung, welche von den Nutzern eingehalten werden müssen. 

Grundsätzlich ist eine Integration eines Security Keys als \ac{SFA} mithilfe einer Azure \ac{AD} möglich. Diese Funktion ist innerhalb der \ac{LSY} allerdings nicht aktiviert, da diese Art der Authentifizierung nicht den Richtlinien des Unternehmens entspricht. Eine Änderung dieser Richtlinien ist möglich, erfordert allerdings einen sehr hohen Aufwand und viel Zeit. Lediglich die Verwendung eines Security Keys als zweiten Faktor für eine \ac{MFA} wird von der \ac{LSY} unterstützt und ist auch mit der Azure \ac{AD} kompatibel (siehe \textbf{\ref{azure-seckey}}). Da der Bearbeitungsraum dieser Arbeit zeitlich limitiert ist, wird aus den genannten Gründen keine \ac{LSY} weite Integration eines Security Keys betrachtet. Dennoch soll eine Integration in einem kleineren Rahmen auf Abteilungsebene getestet werden. Dies soll eine Aussage ermöglichen, ob eine Umstellung der aktuellen Richtlinien sinnvoll ist und eine Integration eines Security Keys in der gesamten \ac{LSY} möglich ist.

\begin{figure}[h]
	\centering 
	\includegraphics[width=0.6\textwidth]{img/abbildungen/azure_seckey.png}
	\captionsetup{format=hang}
	\caption{Security Key als zweiter Faktor in der Azure \ac{AD}} \label{azure-seckey}
\end{figure}

Für diese Testphase wird die Abteilung cGroup Solutions ausgewählt. Diese ist zuständig für das Produkt cFront, welches in \textbf{\ref{cFront-exp}} beschrieben ist. Innerhalb der Abteilung wird aktuell eine passwortbasierte Authentifizierung gegen die Azure \ac{AD} inklusive \ac{MFA} genutzt. Vorteilhaft für die Integration eines Security Keys ist, dass die Abteilung zusätzlich Keycloak für die Identitäts- und Zugriffsverwaltung nutzt \cite{keycloak}. Keycloak ist eine Open-Source-Plattform für Identitäts- und Zugriffsmanagement, die Single Sign-On, Benutzerverwaltung und Sicherheitsfunktionen bietet, um die Authentifizierung und Autorisierung in Anwendungen zu erleichtern. Es wird verwendet, um Benutzer sicher anzumelden, ihre Zugriffsrechte zu verwalten und die Integration mit anderen Identitätsdiensten zu ermöglichen \cite{keycloak}. Somit bietet sich eine neue Schnittstelle für die testweise Integration eines Security Keys an, da Keycloak zwar die Schnittstelle zur Azure \ac{AD} nutzt, allerdings zusätzlich auch eine eigene Nutzerverwaltung ermöglicht. Dies erfordert keine Änderungen der aktuellen Richtlinien oder große technische Umstellungen innerhalb der gesamten \ac{LSY}, sondern ermöglicht eine Integration auf Abteilungsebene.

Die aktuelle Umsetzung der Abteilung zur Authentifizierung ist vereinfacht in \textbf{\ref{current-imp}} abgebildet. Die Applikation kommuniziert nicht direkt mit dem Keycloak-Server, sondern nutzt eine selbst entwickelte Anwendung, welche sich als Keycloak-Client ausgibt. Der Nutzer gibt seine Zugangsdaten bei der Anmeldung an den Keykloak-Client weiter. Dieser wandelt die Zugangsdaten in einen validen JWT-Token um und übergibt diesen an den Keycloak-Server. Der Keycloak-Server validiert die Zugangsdaten gegen die Azure \ac{AD} und authentifiziert den Nutzer, indem er einen oAuth2-Token erstellt und diesen an die Applikation zurückgibt.


\begin{figure}[h]
	\centering 
	\includegraphics[width=1\textwidth]{img/abbildungen/Unknown.png}
	\captionsetup{format=hang}
	\caption{Aktuelle Umsetzung der Abteilung} \label{current-imp}
\end{figure}

\section{Wahl des Security Keys} \label{secwahl}
Für die Umsetzung der passwortlosen Authentifizierung innerhalb der \ac{LSY} wurde ein YubiKey der Series 5 mit NFC gewählt. Dieser wurde in \textbf{\ref{Yubikey}} vorgestellt. Hingewiesen sei an dieser Stelle, dass auch andere Hersteller Security Keys anbieten, welche das FIDO2-Protokoll unterstützen. 

Dazu gehören unter anderem:
\begin{itemize}
    \item \textit{Feitian ePass} des Herstellers FEITIAN Technologies Co., Ltd.
    \item \textit{Titan} des Herstellers Google
    \item \textit{SafeNet eToken} des Herstellers Thales Group
\end{itemize}

Die Wahl des Security Keys richtete sich allerdings unter anderem an der Kompatibilitätsliste \cite{compWin} von Microsoft. Diese listet alle Security Keys auf, welche für eine passwortlose Authentifizierung gegen eine Microsoft Azure \ac{AD} genutzt werden können. Nicht auffindbar in der Liste ist beispielsweise der Google Titan. Dieser unterstützt aktuell nicht FIDO2, sondern lediglich FIDO und \ac{U2F}. Microsoft ist allerdings nicht abwärtskompatibel, was bedeutet, dass der Google Titan nicht für eine passwortlose Authentifizierung gegen das Azure \ac{AD} genutzt werden kann.

Die endgültige Auswahl basiert auf dem bestehenden Bestand eines YubiKeys der Series 5 mit NFC. Dieser ist mit der Azure \ac{AD} nutzbar. Grundsätzlich ist allerdings auch eine Nutzung eines anderen Security Keys möglich, sofern dieser das \ac{FIDO}2-Protokoll unterstützt, in der Kompatibilitätsliste von Microsoft aufgeführt ist und offiziell von der \ac{FIDO} Allianz zertifiziert wurde.

Sollte die Nutzung eines Security Keys innerhalb der \ac{LSY} in Zukunft erweitert werden, so wird eine neue weitreichende Analyse notwendig. Da im Rahmen dieser Arbeit lediglich eine Testphase auf Abteilungsebene stattfindet und die grundsätzliche Funktionsweise der unterschiedlichen Security Keys ähnlich ist, wird auf eine detaillierte Analyse der unterschiedlichen Security Keys verzichtet.

\section{Integration eines YubiKeys in die LSY}

Grundsätzlich bieten sich zwei Möglichkeiten für die Integration eines Security Keys in die Abteilung cGroup Solutions an. Die erste Option wurde bereits \textbf{\ref{current}} beschrieben. Hierbei würde eine Authentifizierung der Nutzer ohne Umwege gegen die Azure \ac{AD} stattfinden, wie in \textbf{\ref{azure-imp}} vereinfacht dargestellt. Diese Option ist nativ mit der Azure \ac{AD} kompatibel und erfordert technisch lediglich eine Anpassung der aktuellen Konfiguration. Dies würde zu einer \ac{LSY} weiten Integration führen, da die Azure \ac{AD} als zentrale Nutzerverwaltung genutzt wird.

\begin{figure}[h]
	\centering 
	\includegraphics[width=0.6\textwidth]{img/abbildungen/azure_umsetzung.png}
	\captionsetup{format=hang}
	\caption{Umsetzungsmöglichkeit mit Azure \ac{AD}} \label{azure-imp}
\end{figure}

Die zweite Option würde Gebrauch von der Schnittstelle des Keycloak-Servers innerhalb der Abteilung cGroup Solutions zu machen. Diese Option würde somit nur die Abteilung betreffen und erfordert eine technische Veränderung der aktuellen Konfiguration des Keycloak-Servers. Hierbei würden nur bestimmten Nutzern die Möglichkeit gegeben werden, sich mithilfe eines Security Keys anzumelden, da der Security Key lediglich vom Keycloak-Server dem Nutzer zugeordnet wird. Also hat diese Option keine Auswirkungen auf die zentrale Nutzerverwaltung. 

Da es sich wie in \textbf{\ref{current}} beschrieben lediglich um eine Testphase handelt, welche nur die Abteilung cGroup Solutions betrifft, wird die zweite Option gewählt. Dies liegt insbesondere an dem hohen organisatorischen Aufwand die aktuellen Richtlinien der \ac{LSY} verändern zu lassen und die technische Umstellung der Azure \ac{AD} zu beantragen. Da die zentrale Nutzerverwaltung ein wichtiger Bestandteil aller Applikationen ist, wäre zudem ein zu hohes Risiko vorhanden, sollten technische Probleme auftreten. 

Um die genannte Option mithilfe von Keycloak umzusetzen, müsste der aktuelle Prozess aus \textbf{\ref{current-imp}} wie folgt angepasst werden:

\begin{figure}[H]
	\centering 
	\includegraphics[width=1\textwidth]{img/abbildungen/keycloak_browser.png}
	\captionsetup{format=hang}
	\caption{Veränderter Keycloak-Login}
\end{figure}

Statt die Anmeldung eines Client zu simulieren, lässt sich mit  Keycloak eine Anmeldung über eine Nutzeroberfläche realisieren. Diese wird dabei selbst von Keycloak gestellt. Dafür wird ein redirect auf die Keycloak Login-Seite durchgeführt. Bei einer erfolgreichen Verifizierung wird der Nutzer zurück auf die Applikation geleitet und vom Keycloak-Server mithilfe eines oAuth2-Tokens authentifiziert. 

Diese Umstellung muss in Keycloak selbst konfiguriert werden. Dafür muss der sog. \textit{Authentication Flow} modifiziert werden. Statt einer Client-Anmeldung erfolgt eine Anmeldung via Browser. Für die Testphase werden zwei Optionen in den Authentication Flow integriert, welcher in \textbf{\ref{auth-flow}} dargestellt sind. Der obere Pfad wird für die passwortlose Authentifizierung genutzt. Dabei gibt der Nutzer zunächst seinen Username ein und authentifiziert sich anschließend mithilfe eines Security Keys und WebAuthn (inklusive CTAP2.1). Der Username wird dabei benötigt, um Nutzern die Möglichkeit zu geben sich mit einem Security Key für mehrere Zugänge zu authentifizieren. Darf ein Nutzer lediglich einen Zugang zur Anwendung besitzen, so kann die Eingabe des Username grundsätzlich entfallen. Der untere Pfad wird für die typische passwortbasierte Authentifizierung genutzt. Diese wird nur integriert da es sich um eine Testphase handelt und dient als Absicherung im Falle von technischen Problemen. Grundsätzlich ist eine Authentifizierung entsprechend dem oberen Pfad ausreichend.

\begin{figure}[H]
	\centering 
	\includegraphics[width=0.7\textwidth]{img/abbildungen/authentication_flow.png}
	\captionsetup{format=hang}
	\caption{Authentication Flow} \label{auth-flow}
\end{figure}

Nach einer erfolgreichen Konfiguration des Authentication Flows, entstehen zwei neue Prozesse für die Anmeldung und für die Registrierung eines Nutzers. Diese werden in den folgenden Grafiken dargestellt:

\begin{figure}[H]
	\centering 
	\includegraphics[width=1\textwidth]{img/abbildungen/register_simplified.png}
	\captionsetup{format=hang}
	\caption{Registrierung (vereinfacht)}
\end{figure}

\begin{figure}[H]
	\centering 
	\includegraphics[width=1\textwidth]{img/abbildungen/login_simplified.png}
	\captionsetup{format=hang}
	\caption{Anmeldung (vereinfacht)}
\end{figure}

Die Grafiken stellen den vereinfachten Ablauf der Registrierung und Anmeldung mithilfe eines Security Keys dar. Eine detaillierte technische Erläuterung und Darstellung der Funktionsweise ist in \textbf{\ref{fido2}} beschrieben. Der entscheidende Unterschied der beiden Prozesse ist allerdings, dass bei der Registrierung lediglich der öffentliche Schlüssel übergeben wird, während bei der Anmeldung der private Schlüssel benötigt wird. Dieser wird allerdings nicht übergeben, sondern signiert eine Login Challenge, welche vom Keycloak-Server generiert wird. Kann der Keycloak-Server die Signatur mithilfe des gespeicherten öffentlichen Schlüssels verifizieren, wird der Nutzer authentifiziert. Sowohl die Registrierung als auch die Anmeldung erfolgen hierbei also nicht über die Anwendung selbst, sondern über den Keycloak-Server und dessen Nutzeroberfläche.

\section{User Feedback}
Um eine Aussage über die Akzeptanz und die Benutzerfreundlichkeit der aufgezeigten Umsetzung zu treffen, wird ein Feedback von den Nutzern der Abteilung cGroup Solutions eingeholt. Um eine wissenschaftliche Aussage zu treffen wird ein Fragebogen erstellt. Es handelt sich dabei um eine Mischform aus einer qualitativen und einer quantitativen Befragung. So wird es ermöglicht eine numerische Auswertung der Antworten zu erhalten, sowie eine qualitative Auswertung der Kommentare. Im Folgenden wird die Durchführung des Fragebogens beschrieben.

\subsection{Rahmen des Feedbacks}
Da zum Zeitpunkt der Erstellung dieser Arbeit die Nutzung keine Umsetzung einer passwortlosen Authentifizierung innerhalb der gesamten \ac{LSY} möglich ist, wird das Feedback auf die Abteilung cGroup Solutions beschränkt. Diese ist zuständig für das in \textbf{\ref{cFront}} beschriebene Produkt cFront, in welchem die passwortlose Authentifizierung testweise implementiert wurde. Die Abteilung besteht aus 15 Personen.

Über einem Zeitraum von zwei Wochen werden alle Mitglieder eingeladen an der Befragung teilzunehmen. Eine Teilnahme ist freiwillig. Die Befragung findet im Büro der Abteilung statt und wird von dem Autor dieser Arbeit durchgeführt. Jeder Teilnehmer wird einzeln und vor Ort befragt. Dies ermöglicht es mit jedem Teilnehmer eine Live-Demonstration durchzuführen. So wird ebenfalls ermöglicht, dass Teilnehmer bereits während der Befragung und der Demonstration Kommentare hinterlassen können. Diese werden auf dem Fragebogen festgehalten und werden für die qualitative Auswertung genutzt werden.

Während der gesamten Demonstration und Befragung werden den Teilnehmern keine Informationen zum FIDO2 Protokoll vermittelt, da sonst die Aussagekraft des Feedbacks verfälscht werden könnte. Ziel ist es den ersten Eindruck aller Teilnehmer zu erhalten, ohne dass diese eine Einführung in die Thematik erhalten. Die Live-Demonstration beinhaltet die Registrierung und Anmeldung mithilfe eines Security Keys, sowie eine Demonstration einer möglichen Anmeldung mithilfe eines Passkeys. Zusätzlich erhalten die Teilnehmer die Möglichkeit den Security Key physisch zu betrachten. Ein detaillierter Verlauf der Demonstration wird im weiteren Verlauf beschrieben.

\subsection{Auswahl der Teilnehmer}
Zur Durchführung des Fragebogens wurden alle Mitglieder des Teams eingeladen, eine Teilnahme war jedoch freiwillig. Drei Mitglieder der Abteilung konnten aufgrund eines Urlaubs nicht an der Befragung teilnehmen. Vor der Durchführung wurden alle Teilnehmer darüber informiert, zu welchem Zweck die Daten für diese Arbeit erhoben werden. Die Befragung fand dabei nicht anonym statt, um einen Austausch zwischen dem Autor und den Teilnehmern zu ermöglichen. Da die Befragung nur die Abteilung der Teilnehmer betrifft, sollten diese somit eine Möglichkeit bekommen, ihre Gedanken zu dem modifiziertem Anmeldevorgang zu teilen.

12 Mitglieder der Abteilung stimmten der Teilnahme an der Befragung zu. Das durchschnittliche Alter der Teilnehmer beträgt 45,4 Jahre. Die genaue Verteilung wird in der folgenden Grafik sichtbar:

\begin{figure}[H]
	\centering 
	\includegraphics[width=0.8\textwidth]{img/abbildungen/chart-2.png}
	\captionsetup{format=hang}
	\caption{Alter der Teilnehmer}
\end{figure}

Dabei ist auffällig, dass die Teilnehmer der Befragung überwiegend der Gruppe 50-60 Jahre zugehörig sind. Auch die Gruppe 20-30 Jahre ist häufig vertreten. Lediglich die Gruppe 30-40 Jahre ist nicht vertreten. Daraus lässt sich schließen, dass die Teilnehmer der Befragung überwiegend eine langjährige Erfahrung in diesem Berufsfeld aufweisen. 

Vor Beginn der Befragung wurden die Teilnehmer gebeten anzugeben, welche Rolle sie innerhalb des Teams einnehmen. Daraus lassen sich drei Gruppen bilden: Developer, Security Specialists und DevOps. Die Verteilung der Teilnehmer auf die beiden Gruppen ist in \textbf{\ref{roles}} dargestellt.

\begin{figure}[h]
	\centering 
	\includegraphics[width=0.7\textwidth]{img/abbildungen/chart_rollen.png}
	\captionsetup{format=hang}
	\caption{Rollen der Teilnehmer} \label{roles}
\end{figure}

\subsection{Inhalt der Demonstration}
Allen Teilnehmern wurde vor der Befragung eine Live-Demonstration der Registrierung und Anmeldung mithilfe eines Security Keys gezeigt. Der Security Key wurde zu Beginn der Demonstration in einen üblichen USB-Slot eines Firmenlaptops eingesteckt und nach der Demonstration an die Teilnehmer übergeben. 

Die Anmeldung/Registrierung ist in mehrere Schritte unterteilt. Zunächst bestätigt der Nutzer, dass er sich mithilfe eines Security Keys anmelden/registrieren möchte:

\begin{figure}[h]
	\centering 
	\includegraphics[width=0.7\textwidth]{img/abbildungen/reg001.png}
	\captionsetup{format=hang}
	\caption{Register}
\end{figure}

Darauf folgt ein Dialogfeld des Browsers, welcher den Nutzer dazu auffordert zu bestätigen, dass der Security Key registriert wird. Dieser Schritt ist einmalig und findet nur bei der Registrierung statt. Ist der Security Key bereits registriert, wird dieser Schritt übersprungen:

\begin{figure}[H]
	\centering 
	\includegraphics[width=0.7\textwidth]{img/abbildungen/reg002.png}
	\captionsetup{format=hang}
	\caption{Dialog des Browsers}
\end{figure}

Nach der Bestätigung des Dialogs muss der Nutzer den PIN des Security Keys eingeben:

\begin{figure}[H]
	\centering 
	\includegraphics[width=0.7\textwidth]{img/abbildungen/reg003.png}
	\captionsetup{format=hang}
	\caption{PIN-Eingabe}
\end{figure}

Ist die richtige PIN eingegeben, erscheint ein letztes Fenster, welches den Nutzer dazu auffordert den Knopf des Security Keys zu drücken. Erst danach ist der Browser dazu autorisiert, sich mithilfe des Security Keys gegen den Keycloak-Server zu registrieren oder anzumelden:

\begin{figure}[h]
	\centering 
	\includegraphics[width=0.7\textwidth]{img/abbildungen/reg004.png}
	\captionsetup{format=hang}
	\caption{Aufforderung des Browsers}
\end{figure}

Sobald der Knopfdruck erfolgt, wird der Nutzer erfolgreich eingeloggt. Diese Informationen wurden den Teilnehmern ebenfalls während der Durchführung des Fragebogens mitgeteilt. 

Nachdem die Registrierung und Anmeldung mithilfe eines Security Keys demonstriert wurde, wurde den Teilnehmern ebenfalls eine mögliche Anmeldung mithilfe eines Passkeys gezeigt. Passkeys werden in \textbf{\ref{passkeys}} beschrieben. Dieser Prozess ist deutlich kürzer und besteht lediglich aus folgendem Dialog:

\begin{figure}[H]
	\centering 
	\includegraphics[width=0.6\textwidth]{img/abbildungen/passkey_demo.png}
	\captionsetup{format=hang}
	\caption{Dialog Passkey}
\end{figure}

Für die Demonstration wurde hierbei ein privates Gerät genutzt (Apple Macbook Air M1), welches mit einem Touch ID Scanner ausgestattet ist.


\subsection{Herleitung der Fragen} \label{questions}
Aufgrund des Ziels der Befragung, eine Aussage über die Akzeptanz und die Benutzerfreundlichkeit einer passwortlosen Authentifizierung zu treffen, werden lediglich Fragen gestellt, die sich auf diese beiden Punkte beziehen. Um eine hohe Teilnahme zu gewährleisten, werden die Fragen möglichst kurz gehalten und nur wenige Fragen gestellt. Die Fragen werden so gestaltet, dass sie dem Teilnehmer die Möglichkeit bietet Kommentare zu hinterlassen oder seine Antwort zu begründen. Die Fragen werden so gestellt, dass sie einfach zu verstehen sind und keine Vorkenntnisse im Bereich der passwortlosen Authentifizierung voraussetzen. Im Folgenden werden die Fragen begründet aufgelistet und erläutert:

\paragraph{Frage 1:}

\begin{quote}
    \textit{Hast du schonmal einen Security Key genutzt?}
\end{quote}
\textbf{Antwortmöglichkeiten:} Ja; Nein; 

Diese Frage leitet sich aus \cite{farke2020you} ab. Die Antwortmöglichkeiten werden im Vergleich aber angepasst und reduziert. Durch die Reduzierung auf zwei Antwortmöglichkeiten wird eine bessere Auswertung ermöglicht. Antworten Teilnehmer mit \textit{Ja}, werden sie gefragt in welchem Kontext sie den Security Key genutzt haben. So lassen sich zusätzliche Informationen über die Nutzungsdauer und den Zweck der Nutzung zu erhalten.

\paragraph{Frage 2:}

\begin{quote}
    \textit{Bist du generell bereit, deine Passwörter durch eine andere Art der Authentifizierung zu ersetzen?}
\end{quote}
\textbf{Antwortmöglichkeiten:} Ja; Nein; 

Diese Frage ergibt sich aus einer Umfrage von Statista, in welcher Teilnehmer gefragt wurden, durch welche Art der Authentifizierung sie das Passwort ersetzen würden. Lediglich 22 \% der Teilnehmer gaben an, dass sie ihr Passwort lieber beibehalten würden \cite{techstat}. Daraus folgt die Annahme, dass eine Vielzahl an Nutzern grundsätzlich dazu bereit wäre, ihr Passwort zu ersetzen. Die Frage soll eine bessere Analyse der folgenden Fragen ermöglichen und zielt auf die Akzeptanz einer passwortlosen Authentifizierung ab.

\paragraph{Frage 3:}

\begin{quote}
    \textit{Benutzt du auf der Arbeit aktuell einen Passwort Manager?}
\end{quote}

\textbf{Antwortmöglichkeiten:} Ja; Nein;

Verwandte Studien zeigen, dass Nutzer eines Passwort Managers teilweise eine geringere Anmeldezeit aufgrund eines Passwort Managers aufweisen (insbesondere bei einer Nutzung von autofill) \cite{farke2020you}. Die Frage soll einen Zusammenhang zwischen der Nutzung eines Passwort Managers und der Einschätzung der Benutzerfreundlichkeit einer passwortlosen Authentifizierung ermöglichen.

\paragraph{Frage 4:}

\begin{quote}
    \textit{Kennst du das FIDO2-Protokoll und weißt du grob wie es funktioniert?}
\end{quote}

\textbf{Antwortmöglichkeiten:} Ja; Nein;

Diese Frage bezieht sich auf die in \textbf{\ref{Yubikey}} aufgeführte Problematik, dass Nutzer lieber Passwörter nutzen, da sie die Funktionsweise und Technologie im Hintergrund besser verstehen. Dieser mögliche Zusammenhang soll betrachtet werden. Antworten Teilnehmer mit \textit{Ja}, werden sie gefragt, ob sie die Funktionsweise des FIDO2-Protokolls erklären können. So lässt sich eine Aussage über die Kenntnisse der Teilnehmer treffen. 

\paragraph{Frage 5:}

\begin{quote}
    \textit{Wie bewertest du die Benutzerfreundlichkeit der Registrierung mithilfe eines Security Keys?}
\end{quote}

\textbf{Antwortmöglichkeiten:} Besser als mit einem Passwort; Gleich gut wie mit einem Passwort; Schlechter als mit einem Passwort;

Diese Frage soll einen Vergleich zwischen der Benutzerfreundlichkeit einer passwortlosen Authentifizierung und einer passwortbasierten Authentifizierung ermöglichen. Aus diesem Grund werden die Antwortmöglichkeiten bewusst so gewählt, dass sie einen Vergleich ermöglichen. Eine generelle Bewertung würde die Auswertung erschweren, da die Teilnehmer unterschiedliche Vergleichswerte wählen könnten.

\paragraph{Frage 6:}

\begin{quote}
    \textit{Wie bewertest du die Benutzerfreundlichkeit der Anmeldung mithilfe eines Security Keys?}
\end{quote}

\textbf{Antwortmöglichkeiten:} Besser als mit einem Passwort und \ac{MFA}; Gleich gut wie mit einem Passwort und \ac{MFA}; Schlechter als mit einem Passwort und \ac{MFA};

Wie auch die vorherige Frage zielt diese Frage auf die Benutzerfreundlichkeit ab. Die Unterteilung in zwei Fragen ergibt sich vor allem aus der Tatsache, dass sich die Registrierung und die Anmeldung, insbesondere bei einer passwortbasierten Authentifizierung, deutlich unterscheiden. Während es sich bei einer Anmeldung lediglich um eine Wissensabfrage handelt, muss bei der Registrierung zunächst ein eigenes Passwort erstellt werden. Dies könnte dazu führen, dass die beiden Abläufe unterschiedlich bewertet werden und somit der Vergleich zur passwortlosen Authentifizierung erschwert wird.

\paragraph{Frage 7:}

\begin{quote}
    \textit{Wärst du dazu bereit einen Security Key für den privaten Gebrauch zu kaufen, wenn der Preis bei ca. 50 € liegt?}
\end{quote}

\textbf{Antwortmöglichkeiten:} Ja; Nein;

Diese Frage zielt auf die Akzeptanz einer passwortlosen Authentifizierung im privaten Kontext ab und basiert auf dem Ergebnis in \textbf{\ref{Yubikey}}. Dort wurde festegestellt, dass der Kaufpreis eines Security Keys ebenfalls eine Hürde für die Nutzung darstellen kann. Als Richtwert für den Kaufpreis wird hierbei der ungefähre Preis eines YubiKeys der Series 5 mit NFC gewählt, da dieser ebenfalls für die Umsetzung genutzt wird. Die Frage soll im Weiteren auch auf für die Nutzung im Unternehmenskontext genutzt werden, da die Akzeptanz im Generellen auch eine Auswirkung auf die Etablierung von Security Keys hat. Eine erhöhte Etablierung kann ebenfalls zu einer breiteren Unterstützung führen.

\paragraph{Frage 8:}

\begin{quote}
    \textit{Hältst du einen Security Key für sicherer als ein Passwort?}
\end{quote}

\textbf{Antwortmöglichkeiten:} Ja; Nein;

Diese Frage basiert auf der in \textbf{\ref{Yubikey}} beschriebenen Annahme, dass Nutzer an der Sicherheit von Security Keys zweifeln, da sie die Funktionsweise der Technologie nicht verstehen. Dies soll im Zusammenhang mit Frage 4 betrachtet werden. Bewusst wird dabei auf die Antwortmöglichkeit \textit{Ich weiß es nicht} verzichtet, da Teilnehmer auf der Basis ihres aktuellen Wissensstands eine intuitive Entscheidung treffen sollen. Dies ermöglicht ebenfalls eine Aussage über die Akzeptanz der Teilnehmer. 

\paragraph{Frage 9:}

\begin{quote}
    \textit{Findest du eine Anmeldung per Passkey besser als eine Anmeldung per Security Key?}
\end{quote}

\textbf{Antwortmöglichkeiten:} Ja; Nein; Gleich;

Diese Frage soll für einen Ausblick genutzt werden, ob eine Anmeldung per Passkey eine Alternative zu einer Anmeldung per Security Key darstellt. In \textbf{\ref{fido2-pros}} wurde ebenfalls deutlich, dass die Benutzerfreundlichkeit weniger von \ac{FIDO}2 abhängig ist, sondern viel mehr vom genutzten Authentifizerungsgerät.  Mithilfe von Kommentaren der Teilnehmer sollen konkrete Vor- und Nachteile der beiden Verfahren in Bezug auf deren Benutzerfreundlichkeit ermittelt werden.


\subsection{Auswertung}
Die Auswertung der Fragebögen bestätigt in vielen Teilen die bereits erarbeiteten Annahmen aus \textbf{\ref{Yubikey}}. Lediglich zwei der Teilnehmer geben an, dass sie bereits einen Security Key genutzt haben. Dies allerdings nur testweise und nicht im alltäglichen Gebrauch. Die restlichen Teilnehmer geben an, dass sie noch keinen Security Key genutzt haben bzw. lediglich einen gesehen haben. Dies bestätigt, dass die Nutzung von Security Keys aktuell noch nicht weit verbreitet ist und Passwörter weiterhin die dominierende Methode der Authentifizierung darstellen.

Alle Teilnehmer geben an, dass sie dazu bereit wären ihr Passwort durch eine andere Art der Authentifizierung zu ersetzen. Dies übertrifft das Ergebnis aus \cite{techstat}. Dies kann daran liegen, dass alle Teilnehmer in einem sehr technischen Kontext arbeiten und somit eine höhere Akzeptanz für neue Technologien aufweisen und ein höheres Bewusstsein für Sicherheit innerhalb der Informatik aufweisen. 

\begin{figure}[H]
	\centering 
	\includegraphics[width=0.7\textwidth]{img/abbildungen/chart_anmeldung_register.png}
	\captionsetup{format=hang}
	\caption{Auswertung Frage 5 \& 6}
\end{figure}

Bei der Bewertung der Registrierung und der Anmeldung mithilfe eine Security Keys sind deutliche Unterschiede sichtbar. Die deutliche Mehrheit der Teilnehmer bevorzugte die Registrierung per Security Key gegenüber der passwortbasierten Alternative. Keiner der Teilnehmer fand die Registrierung im Vergleich schlechter. Bei der Anmeldung hingegen ist ein ausgeglicheneres Ergebnis sichtbar. Im Vergleich geben drei der Teilnehmer an, dass sie Anmeldung schlechter finden als die aktuelle Alternative mithilfe eines Passwortes und \ac{MFA}. Es wird also deutlich, dass die beiden Abläufe der Registrierung und der Anmeldung differenziert betrachtet werden müssen. 
Eine Abhängigkeit zwischen der Nutzung eines Passwort Managers und der Bewertung der Benutzerfreundlichkeit lässt sich nicht feststellen, da lediglich drei Teilnehmer keinen Passwort Manager nutzen und diese sehr verschiedene Bewertungen abgeben. Eine aussagekräftige Auswertung ist somit nicht möglich. 


Zehn der Teilnehmer geben an, dass sie nicht bereit wären 50 € für einen Security Key auszugeben. Dies bestätigt die Annahme aus \textbf{\ref{Yubikey}}, dass der Preis eine Hürde für die Nutzung und Etablierung darstellen kann. Eine vermehrte Nutzung im privaten Kontext würde zu mehr Akzeptanz führen, da die Teilnehmer bereits mit der Technologie vertraut sind. 

Bis auf zwei Teilnehmer wird die Nutzung von Security Keys sicherer eingeschätzt als die Nutzung von Passwörtern. Da lediglich zwei Nutzer die Nutzung als weniger sicher betrachten lässt sich keine Aussage über einen Zusammenhang zwischen der Kenntnis des FIDO2-Protokolls und der Einschätzung der Sicherheit treffen.

Zwölf der Teilnehmer geben an eine Anmeldung per Passkey besser zu finden als eine Anmeldung per Security Key. 

Aus den Kommentaren lassen sich ebenfalls einige Erkenntnisse ziehen:

\begin{itemize}
    \item Ein Teilnehmer stufte die Registrierung und Anmeldung als \glqq\textit{sehr aufregend}\grqq, da es etwas neues ist und begründete so seine positive Bewertung. Dieses Beispiel zeigt auf, dass die Gewöhnung an eine neue Technologie nicht zwangsweise negativ ist, sondern auch positiv bewertet werden kann.
    \item Mehrere Teilnehmer kritisierten die Notwendigkeit, den Security Key immer dabei haben zu müssen und somit spontane Logins nicht möglich sind. Dies deckt sich mit den Ergebnissen aus \textbf{\ref{Yubikey}}.
    \item Daraus resultiert auch der Kritikpunkt, dass zusätzliche Hardware verloren gehen kann. Dies ist ein weiterer Kritikpunkt, welcher bereits in \textbf{\ref{Yubikey}} aufgeführt wurde.
    \item Mehrere Teilnehmer wiesen darauf hin, dass sie ihr Handy und somit ihre Authenticator App immer dabei haben. Selbst, wenn sie den Prozess des Anmeldens mithilfe eines Security Keys als besser bewerten, würden sie weiterhin die Nutzung eines Passwortes mit \ac{MFA} bevorzugen. 
    \item Ein neuer Punkt, der in den Kommentaren aufgeführt wurde, ist dass \ac{SSO} ein wichtiger Faktor ist. Da somit eine geringere Abfrage der Passwörter gegeben ist und auch weniger Passwörter erstellt werden müssen. Dies wirkt sich auch auf die Einschätzung der Benutzerfreundlichkeit aus.
    \item Die Mehrheit der Teilnehmer ist der Meinung, dass sie die Benutzerfreundlichkeit der Security Keys erst nach einer längeren Testphase bewerten können.
    \item Ein Kritikpunkt der Teilnehmer am Anmeldevorgang mithilfe eines Security Keys ist, dass die Eingabe der PIN und des Benutzernamens zu viel Aufwand sind. Dadurch bewerteten sie die Alternative als gleich oder schlechter im Vergleich zu der aktuellen Lösung. Sie wünschen sich eine Lösung, bei dem der Security Key lediglich eingesteckt und gedrückt werden muss.
    \item Ein Teilnehmer begründet seine Antwort bezogen auf die Passkeys damit, dass er zusätzliche Hardware für sicherer hält und sich bei der Nutzung von Passkeys nicht sicher ist, ob diese wirklich nur auf dem Gerät gespeichert werden.
    \item Einige Teilnehmer erläuterten, dass sie die Anmeldung per Security Key benutzerfreundlicher als ein Passwort mit \ac{MFA} finden, allerdings nicht besser als lediglich einem Passwort.
    \item Ebenfalls wurde die mechanische Belastung des Security Keys und des USB-Ports genannt. Beide könnten durch die Nutzung beschädigt werden. Auch dieser Punkt wurde bereits in \textbf{\ref{Yubikey}} aufgeführt.
    \item Auch der Preis wurde häufig als Kritikpunkt genannt. Ergänzend erwähnte allerdings ein Teilnehmer, dass er den Preis bezahlen würde, wenn sich das Verfahren weiter etabliert. Ein anderer Teilnehmer erwähnte, dass er zum aktuellen Zeitpunkt maximal 20 € für einen Security Key ausgeben würde.
    \item Zur reinen Benutzerfreundlichkeit merkte ein Teilnehmer an, dass ein schlechtes und einfach gewähltes Passwort deutlich benutzerfreundlicher sei, wenn man den Faktor der Sicherheit nicht betrachtet.
\end{itemize}


\section{Wirtschaftlichkeit}
Ein entscheidender Faktor für den Einsatz einer passwortlosen Authentifizierung mithilfe eines Security Keys ist die Wirtschaftlichkeit. Im Folgenden soll eine Analyse der Wirtschaftlichkeit in Bezug auf die Abteilung cGroup Solutions durchgeführt werden.

Betrachtet man die Teamgröße von 15 Personen und geht von den aktuellen Kosten eines YubiKeys aus (50 €) so ergibt sich ein Gesamtpreis von 750 €. Dieser Preis ist einmalig und muss nicht wiederholt werden. Aufgrund der geringen Backup-Möglichkeiten eines Security Keys sollte allerdings ein Backup-Key pro Nutzer angeschafft werden. Dieser kann im Falle eines Verlustes des ersten Keys genutzt werden. Dies erhöht die Kosten auf 1500 €.
Lässt man alternativ weiterhin eine Anmeldung per Passwort zu, würden 750 € entfallen, allerdings wäre der eigentliche Zweck der Anschaffung nicht mehr erfüllt.
Sollte ein Security Key verloren oder kaputtgehen muss dieser ebenfalls ersetzt werden.
Zusätzlich zu den Materialkosten müssen die Kosten für die Implementierung betrachtet werden. Die Migration von Passwörtern auf Security Keys muss von erfahrenen Entwicklern und Architekten durchgeführt werden. Die genauen Kosten dafür lassen sich allerdings nicht definieren und sind stark abhängig von der gewünschten Implementierung. 

Sollte eine Anschaffung für das gesamte Unternehmen \ac{LSY} erfolgen so würde sich der Preis auf 280.000 € belaufen. Dies ergibt sich aus der aktuellen Mitarbeiterzahl von 2.800 und den Kosten eines Security Keys von 50 € sowie einem Backup-Key pro Nutzer.

Im Gegenzug muss ein möglicher Kostenvorteil eingerechnet werden. Da Passwörter wie in \textbf{\ref{pw-auth}} beschrieben eine der größten Schwachstellen darstellen, sind sie ein grundlegender Faktor für erfolgreiche Angriffe. Würde eine Nutzung der Security Keys dies verhindern oder eindämmen, so würde dies zu einer Kostenreduktion führen. Auch hier lassen sich allerdings nur schwierig konkrete Zahlen nennen. Diese sind unter anderem abhängig von der Schwere und Art des Angriffs. Einen Richtwert liefert der \textit{Cost of a Data Breach Report 2023} von IBM \cite{databreach}. Dort werden die durchschnittlichen Kosten eines Data Breach für bestimmte Regionen aufgezählt. Der Wert für die Region Deutschland liegt bei 4,67 Millionen US-Dollar \cite{databreach}. Daraus lässt sich folgern, dass ein erfolgreicher Angriff auf die \ac{LSY} zu einem deutlich höheren Kostenaufwand führen kann, als die Anschaffung von Security Keys.


\input{content/c5.tex}

\printbibliography[title=Literaturverzeichnis]
% Der Anhang beginnt hier - jedes Kapitel wird alphabetisch aufgezählt. (Anhang A, B usw.)
%\appendix
%\ihead{\appendixname~\thechapter} % Neue Header-Definition


% appendix.tex einziehen
%\input{appendix}


\end{document}
