%
%   Prof. Dr. Julian Reichwald
%   auf Basis einer Vorlage von Prof. Dr. Jörg Baumgart
%   DHBW Mannheim
%
%
%	ACHTUNG: Für das Erstellen des Literaturverzeichnisses wird das modernere Paket biblatex
%			 in Kombination mit biber verwendet -- nicht mehr das ältere BibTex!
% 			 Bitte stellen Sie ggf. Ihre TeX-Umgebung
% 			 entsprechend ein (z.B. TeXStudio: Einstellungen --> Erzeugen --> Standard Bibliographieprogramm: biber)
%
% LTeX: language=de-DE

\documentclass[
	12pt,
	BCOR=5mm,
	DIV=10,
	headinclude=on,	
	footinclude=off,
	parskip=half,
	bibliography=totoc,
	listof=entryprefix,
	toc=listof,
	pointlessnumbers
	]{scrreprt}

%	Konfigurationsdatei einziehen
\input{config}

\begin{document}



\TitelDerArbeit{Analyse und Integration einer passwortlosen Authentifizierung im Unternehmenskontext}
\AutorDerArbeit{Luka Tsipitsoudis}
\Firma{Lufthansa Systems GmbH \& Co. KG}
\Kurs{TINF20CS1}

\definecolor{tabdavid}{RGB}{213,229,255}

\pagenumbering{Alph}
\setcounter{page}{1} 

\begin{titlepage}
    \begin{minipage}{\textwidth}
            \vspace{-2cm}
            \noindent
            \includegraphics[scale=0.035]{img/Uzletiforditas.hu-Lufthansa.png} 
             \hfill   
             \includegraphics{img/logo.jpg}
    \end{minipage}
    
    \vspace{4em}
    \sffamily
    \begin{center}
        \textsf{\large{}Duale Hochschule Baden-W\"urttemberg\\[1.5mm] Mannheim}\\[2em]
        \textsf{\textbf{\Large{}Bachelorarbeit}}\\[3mm]
        \vspace{5em}
        \textsf{\textbf{\Large{\DerTitelDerArbeit}}} \\[1.5cm]
        \vspace{5em}
        \textsf{\textbf{Studiengang Cyber Security}\\[3mm]}
        
        \vspace{3em}
    \vfill
    
    \begin{minipage}{\textwidth}
    
    \begin{tabbing}
        Wissenschaftlicher Betreuer: \hspace{0.85cm}\=\kill
        Verfasser: \> \DerAutorDerArbeit \\[1.5mm]
        Matrikelnummer: \> 4110112 \\[1.5mm]
        Kurs: \> \DieKursbezeichnung \\[1.5mm]
        Bearbeitungszeitraum: \> 06.06.2023 -- 29.08.2023\\ [1.5mm]
        Abgabedatum: \> 29.08.2023\\ [1.5mm]
        Betreuer: \> Stefan Köster\\
        % Ausbildungsfirma: \> \DerNameDerFirma  \\[1.5mm]
        % Betrieblicher Betreuer: \> Oliver Grimm \\
    \end{tabbing}
    
    % \vspace{2cm}
    % Unterschrift Betreuer:  \> \_\_\_\_\_\_\_\_\_\_\_\_\_\_\_\_\_\_ 
    \end{minipage}
    
    \end{center}
    
    \end{titlepage}

\pagenumbering{roman} % Römische Seitennummerierung
\normalfont

%--------------------------------
% Verzeichnisse - nicht benötige Verzeichnisse bitte auskommentieren / löschen.
%--------------------------------

% Ehrenwörtliche Erklärung ewerkl.tex einziehen
\input{ewerkl1.tex}
% \input{ewerkl2.tex}

%   Sperrvermerk
% \input{nondisclosurenotice}

% LTeX: language=de-DE

\chapter*{Abstract}

\vspace{-2em}

\paragraph*{Englisch}
Passwords are considered to be the most common type of authentication. Despite their high establishment, they have a large number of vulnerabilities. For this reason, some passwordless alternatives for authentication have already been developed. One of these alternatives is the FIDO2 standard. It allows users to log in to online services using an authentication device. Frequently used authentication devices are security keys. These are an external piece of hardware that is only in the possession of the user. The FIDO2 standard is based on asymmetric cryptography, which allows authentication to be more secure than password-based authentication. This is especially due to the fact that FIDO2 credentials cannot be guessed and cannot be affected by data leaks or phishing emails. This thesis deals with the factors of security and usability of FIDO2 and with the integration of FIDO2 into Lufthansa Systems GmbH \& Co. KG.

\paragraph*{Deutsch}
Passwörter gelten seit langer Zeit als die gängigste Art der Authentifizierung. Trotz ihrer hohen Etablierung weisen sie eine Vielzahl an Schwachstellen auf. Aus diesem Grund wurden bereits einige passwortlose Alternativen zur Authentifizierung entwickelt. Eine dieser Alternativen ist der FIDO2 Standard. Er bietet Nutzern die Möglichkeit sich mithilfe eines Authentifizierungsgerätes bei Online-Diensten anzumelden. Häufig genutzte Authentifizierungsgeräte sind dabei Security Keys. Bei diesen handelt es sich um eine externe Hardware, welche sich nur im Besitz des Nutzers befindet. Der FIDO2 Standard basiert auf asymmetrischer Kryptografie, was eine Authentifizierung sicherer gestalten kann als eine passwortbasierte Authentifizierung. Dies liegt insbesondere daran, dass sich FIDO2 Zugangsdaten nicht erraten lassen und nicht von Datenlecks oder Phishing E-Mails betroffen seien können. Diese Arbeit beschäftigt sich mit den Faktoren der Sicherheit und Benutzerfreundlichkeit von FIDO2 und mit der Integration von FIDO2 in die Lufthansa Systems GmbH \& Co. KG.

%	Inhaltsverzeichnis
\tableofcontents



%	Abbildungsverzeichnis
\listoffigures

%	Tabellenverzeichnis
\listoftables

%	Listingsverzeichnis
%\lstlistoflistings

% 	Algorithmenverzeichnis
%\listofalgorithms

% 	Abkürzungsverzeichnis (siehe Datei acronyms.tex!)
\clearpage
\chapter*{Abkürzungsverzeichnis}	
\addcontentsline{toc}{chapter}{Abkürzungsverzeichnis}


\begin{acronym}[RDBMS]
	\acro{LSY}{Lufthansa Systems GmbH \& Co. KG}
	\acro{FIDO}{Fast Identity Online}
	\acro{W3C}{World Wide Web Consortium}
	\acro{SFA}{Single-Factor Authentication}
	\acro{MFA}{Multi-Factor Authentication}
	\acro{CTAP2}{Client-to-Authenticator Protocol 2}
	\acro{NFC}{Near Field Communication}
	\acro{ITU-T}{International Telecommunication Union Telecommunication Standardization Sector}
\end{acronym}

\ohead{Acronyms} % Neue Header-Definition


%--------------------------------
% Start des Textteils der Arbeit
%--------------------------------
\clearpage
\ihead{\chaptername~\thechapter} % Neue Header-Definition (inner header)
\ohead{\headmark} % Neue Header-Definition (outer header)
\pagenumbering{arabic}  % Arabische Seitenzahlen



\clearpage
\pagenumbering{arabic}
\setcounter{page}{1}
%	Literaturverzeichnis
\clearpage
\ihead{}

\cleardoublepage

\chapter{Problemstellung \& Ziel der Arbeit}

\chapter{Grundlagen}

\section{Einführung in cFront}

\section{Passwortbasierte Authentifizierung}

\section{Passwortlose Authentifizierung}

\chapter{Umsetzung}

\section{Aktueller Stand der LSY}

\section{Integration eines Yubikeys in die LSY}

\section{Nutzung des passwortlosen Verfahrens im Unternehmenskontext}

\section{User Feedback}

\section{Zeitmessung}

\section{Nutzung des passwortlosen Verfahrens im privaten Kontext}


\chapter{Fazit \& Empfehlung}
In Kapitel xy wurden insbesondere drei Ziele der Arbeit definiert. Diese werden im Folgenden bewertet. 
Die Ergebnisse der Arbeit machen deutlich, welche Schwachstellen und Angriffsvektoren aufzeigen. Aus diesem Grund existiert jedoch nicht bloß eine Alternative. Es gibt verschiedene passwortlose Ansätze, welche sich für verschiedene Anwendungsfälle eignen. Jeder Ansatz hat dabei seine Vor- und Nachteile. Deutlich wird jedoch, dass das FIDO2-Projekt zu den meist untestützten und am weitesten verbreiteten Ansätzen gehört. Dies liegt auch an der gebotenen Vielfalt, da FIDO2 nicht nur Security Keys, sondern beispielsweise auch Passkeys unterstützt.
Im Bezug auf Sicherheit ist das FIDO2-Protokoll eine erhebliche Verbesserung gegenüber der klassischen Passwortauthentifizierung. Da FIDO2 auf öffentliche/private Schlüssel basiert fallen die meisten Angriffsvektoren der klassischen Passwortauthentifizierung weg. 

Aus diesen Gründen empfiehlt sich die Integration von FIDO2 als Alternative ebenfalls für den Unternehmenskontext. Für die Nutzung von Security Keys hingegegen lässt sich keine eindeutige Empfehlung auf den gegebenen Kontext der \ac{LSY} aussprechen. Dies geht vor allem aus den Umsetzungsmöglichkeiten und der erarbeiteten Benutzerfreundlichkeit hervor. Aktuell ist eine FIDO2 Authentifizierung mit Hilfe eines Security Keys noch nicht ausreichend etabliert, um eine gesamte Umstellung vornehmen zu können. Nicht alle Dienste ermöglichen eine FIDO2 Authentifizierung. Zum aktuellen Zeitpunkt eignet sich die Nutzung von Security Keys lediglich für eine \ac{MFA}. Das Ergebnis dieser Arbeit ist allerdings, dass sich die Nutzung von Security Keys insbesondere für eine \ac{SFA} eignet.

Dies entspricht nicht den Richtlinien \ac{SFA}. Solange allerdings keine weitreichende Unterstützung von FIDO2 erfolgt, wird keine Änderung dieser Richtlinie empfohlen. Die Nutzung von Security Keys als zusätzlicher Faktor wird aus wirtschaftlichen Gründen nicht empfohlen. In diesem Fall ist die Nutzung einer Authenticator App besser geeignet.

Grundsätzlich ist eine Nutzung von FIDO2 als Alternative zur klassischen Passwortauthentifizierung zu empfehlen. Sobald eine ausschließliche Nutzung ermöglicht wird lassen siche erhebliche Vorteile für die Sicherheit erzielen. Lediglich die Nutzung von Security Keys ist nicht zweifelsfrei zu empfehlen. In dieser Arbeit werden mehrere Kritikpunkte an der Benutzerfreundlichkeit aufgezeigt. Fragwürdig ist, ob diese Kritikpunkte nach einer erweiterten Gewöhnungsphase noch bestehen. Eine Lösung könnte die Nutzung von Passkeys darstellen. Diese sind allerdings noch nicht weitreichend verbreitet.

Daraus folgt die Empfehlung für die \ac{LSY} nicht direkt auf FIDO2 in Kombination mit Security Keys zu setzen. Stattdessen sollte das Bewusstsein für passwortlose Alternativen erweitert werden. Die Offenheit für Alternativen sollte ebenfalls gefördert. Zusätzlich sollte die Etablierung von FIDO2 weiter beobachtet werden. Sobald eine ausschließliche Nutzung möglich ist, sollte eine Umstellung erfolgen.

\chapter{Ausblick Passkeys}

\printbibliography[title=Literaturverzeichnis]
% Der Anhang beginnt hier - jedes Kapitel wird alphabetisch aufgezählt. (Anhang A, B usw.)
%\appendix
%\ihead{\appendixname~\thechapter} % Neue Header-Definition


% appendix.tex einziehen
%\input{appendix}


\end{document}
